\documentclass[a4paper]{extreport}
\usepackage[14pt]{extsizes}
\usepackage{fontspec}
\defaultfontfeatures{Ligatures={TeX},Renderer=Basic} 
\setmainfont[Ligatures={TeX,Historic}]{Times New Roman}
\usepackage[english, russian]{babel}
\usepackage{setspace}
\usepackage{float}
\usepackage{indentfirst}
\usepackage{amsmath}
\usepackage{graphicx}
\usepackage{caption}
\usepackage{subcaption}
\usepackage{array}
\usepackage{titlesec}
\usepackage{appendix}
\usepackage[footnote]{acronym}
\usepackage[top=2cm, bottom=2cm, left=3cm, right=1cm, nofoot, nohead]{geometry}
\usepackage{titlesec}
\usepackage{tocloft}
\usepackage{lastpage}
\usepackage{etoolbox}
\usepackage{fancyhdr}

\pagestyle{empty} 
\titleformat{\chapter}[display]
{\filcenter}
{{\large\textbf{\MakeUppercase{\chaptertitlename} \thechapter}}}
{5pt}
{\bfseries}{}

\titleformat{\section}
{\normalsize\bfseries}
{\thesection}
{1em}{}

\titleformat{\subsection}
{\normalsize\bfseries}
{\thesubsection}
{1em}{}

% Настройка вертикальных и горизонтальных отступов
\titlespacing*{\chapter}{0pt}{-30pt}{8pt}
\titlespacing*{\section}{\parindent}{*4}{*1}
\titlespacing*{\subsection}{\parindent}{*4}{*1}

\newcommand{\likechapterheading}[1]{ 
	\begin{center}
		\textbf{\MakeUppercase{#1}}
	\end{center}
}

\makeatletter
\renewcommand{\@dotsep}{2}
\newcommand{\l@likechapter}[2]{{\@dottedtocline{0}{0pt}{0pt}{\bfseries #1}{\bfseries #2}}}
\makeatother

\newcommand{\likechapter}[1]{    
	\likechapterheading{#1}    
	\addcontentsline{toc}{likechapter}{\MakeUppercase{#1}}}

\renewcommand{\cfttoctitlefont}{\hspace{0.38\textwidth} \bfseries\MakeUppercase}
\renewcommand{\cftbeforetoctitleskip}{-1em}
\renewcommand{\cftchapfont}{\normalsize\bfseries \MakeUppercase{\chaptername} }
\renewcommand{\cftbeforechapskip}{1em}
\renewcommand{\cftparskip}{-1mm}
\renewcommand{\cftdotsep}{1}
\setcounter{tocdepth}{2}

\linespread{1.1}
\setlength{\parindent}{1.25cm}
\bibliographystyle{plain}
\renewcommand{\contentsname}{Оглавление}
\counterwithin{figure}{chapter}
\setlength{\footskip}{25pt}

% для en версии заменить russian на english %
\addto\captionsrussian{\renewcommand{\bibname}{}}
\patchcmd{\thebibliography}{\chapter*}{}{}{}
\patchcmd{\thebibliography}{\chapter*{\refname}}{}{}{}


\newcommand{\likechapterheeading}[1]{ 
	\begin{flushright}
		\textbf{\MakeUppercase{#1}}
	\end{flushright}
}

\makeatletter
\renewcommand{\@dotsep}{2}
\newcommand{\l@likeechapter}[2]{{\@dottedtocline{0}{0pt}{0pt}{\bfseries #1}{\bfseries #2}}}
\makeatother

\newcommand{\likeechapter}[1]{    
	\likechapterheeading{#1}    
	\addcontentsline{toc}{likechapter}{\MakeUppercase{#1}}}


\pretocmd{\chapter}{\addtocounter{totfigures}{\value{figure}}}{}{}
\pretocmd{\chapter}{\addtocounter{tottables}{\value{table}}}{}{}

\newcounter{totreferences}
\pretocmd{\bibitem}{\addtocounter{totreferences}{1}}{}{}

\newcounter{totfigures}
\newcounter{tottables}
\makeatletter
\AtEndDocument{%
	\addtocounter{totfigures}{\value{figure}}%
	\addtocounter{tottables}{\value{table}}%
	\immediate\write\@mainaux{%
		\string\gdef\string\totfig{\number\value{totfigures}}%
		\string\gdef\string\tottab{\number\value{tottables}}%
		\string\gdef\string\totref{\number\value{totreferences}}%    
	}%
}
\makeatother

\begin{document}
	\begin{center}
		\textbf{БЕЛОРУССКИЙ ГОСУДАРСТВЕННЫЙ УНИВЕРСИТЕТ\\}
	\end{center}
	\begin{minipage}{0.4\textwidth}
		\textbf{Факультет}\\
		\underline{Механико-математический}\\
	\end{minipage}
	\hfill
	\begin{minipage}{0.4\textwidth}
		\textbf{Кафедра}\\
		\underline{Теоретической и прикладной}\\
		\underline{механики}
	\end{minipage}
	\\
	\begin{flushleft}
		Заведующий кафедрой,\\
		профессор М. А. Журавков\\
		$\underline{\hspace{6cm}}$\\
		<<$\underline{\hspace{1cm}}$>>$\underline{\hspace{3cm}}$ 2023 г.
	\end{flushleft}
	\begin{center}
		\textbf{\MakeUppercase{задание на дипломную работу}}
	\end{center}
	\begin{flushleft}
		\textbf{Студенту:} Шевелёву Дмитрию Юрьевичу\\
		\textbf{Тема дипломной работы:} <<Влияние вихрегенераторов в турбулентном пограничном слое на локальное трение и перенос>>\\
		\textbf{Руководитель дипломной работы:} кандидат физико-математических наук, доцент Чорный Андрей Дмитриевич\\
		\textbf{Постановка задачи на дипломную работу:}\\
		% Сюда написать)) %
		\textbf{Рекомендуемые источники информации:}\\
		% Пару источников %
		\textbf{Краткое обоснование актуальности темы дипломной работы:}\\
		% Небольшой абзац %
		\textbf{Форма презентации дипломной работы:}\\
		Презентация Power Point\\
		\textbf{График выполнения дипломной работы:}\\
		% Надо подумать над сроками и оформить пару отрезков времени %
		05.12.2022 - 07.02.2023 --- Изучение теоретического материала;\\
		08.02.2023 - 03.03.2023 --- работа над геометрией, создание, тестирование и улучшение сеточной модели;\\
		06.03.2023 - 21.03.2023 --- проведение тестовых расчётов методом RANS и устранение неточностей модели;\\
		22.03.2023 - 14.04.2023 --- вычисление методом LES;\\
		17.04.2023 - 10.05.2023 --- анализ полученных данных и изучение влияния вихрегенераторов на локальное трение и перенос.\\
		$\underline{\hspace{\textwidth}}$
		Предоставление чернового варианта работы: 24.04.2023 г.\\
		Предзащита: 24.05.2023 г.\\
		Защита: 14.06.2023 г.\\
	\end{flushleft}
	\begin{minipage}{0.5\textwidth}
		\textbf{Руководитель дипломной работы:}\\
		\textbf{Задание к исполнению принял:}\\
	\end{minipage}
	\hfill
	\begin{minipage}{0.5\textwidth}
		$\underline{\hspace{4cm}}$ А. Д. Чорный\\
		$\underline{\hspace{4cm}}$ Д. Ю. Шевелёв\\
	\end{minipage}
	\\
	<<$\underline{\hspace{1cm}}$>>$\underline{\hspace{3cm}}$ 2023 г.\\
\end{document}