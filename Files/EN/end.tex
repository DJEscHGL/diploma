\begin{center}
	\Large\textbf{Conclusion}
\end{center}

In this paper, various methods for modeling turbulent flows, their advantages and disadvantages are studied, and performance is also evaluated. In addition, methods for constructing a grid model were studied and an optimal scheme for working on it was created.

As a result of the channel simulation in Ansys Fluent, the influence on the boundary layer was estimated by modeling large eddies. Due to the placement of the vortex generator across, a turbulent mode of motion was formed. The presence of a vortex generator affects the change in the velocity characteristics of the flow in the region of the boundary layer following it downstream, which leads to the appearance of large-scale vortex structures in this region. These vortex structures significantly affected local friction and transport. These changes are reflected in a sharp change in the friction coefficient by 2-3 times. However, after this, a decrease in the coefficient to the values up to the vortex generator is observed.

Thanks to this work, the influence of the vortex generator was established. In subsequent works, further study is possible, but with other conditions and different types of vortex generators