
В данной работе изучены различные методы моделирования турбулентных течений, их преимущества и недостатки, а также оценена производительность. Помимо этого изучены способы построения сеточной модели и создана оптимальная схема по работе над ней.

В результате проведённого моделирования канала в Ansys Fluent методом моделирования крупных вихрей было оценено влияние на пограничный слой. Благодаря размещению вихрегенератора поперёк образовался турбулентный режим движения. Эти вихревые структуры значительно повлияли на локальное трение и перенос.  