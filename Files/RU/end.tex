
В данной работе изучены различные методы моделирования турбулентных течений, их преимущества и недостатки, а также оценена производительность. Помимо этого изучены способы построения сеточной модели и создана оптимальная схема по работе над ней.

Расчет проводился методом моделирования крупных вихрей(LES) и подсеточной моделью WALE с коэффициентом $C_w = 0.325$. Эта комбинация позволила достаточно точно рассчитать турбулентное состояние пограничного слоя. Кроме того использовалась оптимальная схема построения сеточной модели, что сократило время генерации в несколько часов.

В результате проведенного моделирования канала в Ansys Fluent было оценено влияние вихрегенераторов на пограничный слой. Их наличие изменило скоростные характеристика потока за препятствием. Благодаря размещению вихрегенератора поперек образовался турбулентный режим движения. Появившиеся вихревые структуры воздействовали на стенки канала.

До изменений в потоке коэффициент трения составлял 0.4. После установки вихрегенератора значительное изменение в коэффициенте трения оказалось в правой части канала($z = -31$ мм), он вырос в 4.5 раза. В левой($z = 31$ мм) -- увеличение в 3 раза. А по центру канала($z = 0$ мм) в 2.25 раза. Но уже после вихрегенератора значение коэффициента снижалось, стремясь к значениям до изменения в канале.

В последующих исследованиях возможно изучение с другими условиями движения потока, количеством и вариациями вихрегенераторов.