
В данной работе изучены различные методы моделирования турбулентных течений, их преимущества и недостатки, а также оценена производительность. Помимо этого изучены способы построения сеточной модели и создана оптимальная схема по работе над ней.

В результате проведённого моделирования канала в Ansys Fluent методом моделирования крупных вихрей было оценено влияние на пограничный слой. Их наличие повлияло на скоростные характеристика потока за препятствием. Благодаря размещению вихрегенератора поперёк образовался турбулентный режим движения. Эти вихревые структуры значительно повлияли на локальное трение и перенос. Коэффициент трения достиг значения в 4.5 раза. Но уже после вихрегенератора коэффициент снижался, стремясь к значениям до изменения в канале.

В последующих исследованиях возможно изучение с другими условиями и вариациями вихрегенераторов.