\section{Обработка данных}
	В результате вычислений, которые заняли примерно месяц, были полученные $.dat$ файлы. Каждый из этих массивов данных соответствует определённому временному шагу. 
	
	После окончания вычислений был автоматически сгенерирован график, благодаря которому можно оценить точность полученных результатов на данной сеточной модели. Она достигла значения в $10^{-3}$.
	\begin{figure}[H]
		\centering
		\includegraphics[width=1\linewidth]{../Assets/scaledResiduals}
		\caption{График сходимости}
		\label{fig:scaledresiduals}
	\end{figure}
	
	Для последующей обработки и на их основе построение графиков использовалось встроенное средство Ansys CFD-Post. Это программа для обработки результатов расчетов, которая обладает широким набором всех необходимых инструментов: создание анимаций, графиков, контурных распределений параметров потока, векторных полей, линий тока, объемного рендеринга, отчетов по шаблонам и многого другого. Для анализа влияния вихрегенераторов на локальное трение и перенос использовались различные сечения в разных частях канала. Некоторые из них представлены на рисунке \ref{fig:planesforanalysis}. 
	\begin{figure}[H]
		\centering
		\includegraphics[width=0.9\linewidth]{../Assets/1}
		\caption{Сечения для анализа}
		\label{fig:planesforanalysis}
	\end{figure}
	
	Начало координатных осей модели расположено в центре входа канала. Рассматривались сечения на разных высотах вдоль и различных расстояниях от входа в канал. На таблице ниже представлен перечень сечений, расположение и их описание. Изображения в приложениях имеют названия исходя из таблицы.
	\begin{table}[H]
		\begin{center}
			\begin{tabular}{|c|c|c|c|}
				\hline
				Подпись & Плоскость & Расположение, мм & Описание\\
				\hline
				PlaneXY & XY & Z = 0 & вдоль всего канала\\
				\hline
				PlaneYZ300 & YZ & X = 300 & перед препятствием\\
				\hline
				PlaneYZ340 & YZ & X = 340 & за проволокой\\
				\hline
				PlaneYZ400 & YZ & X = 400 & на входе в прямой участок\\
				\hline
				PlaneYZ600 & YZ & X = 600 & далее по каналу\\
				\hline
				PlaneYZ1400 & YZ & X = 1400 & на выходе из канала\\
				\hline
				PlaneXZ0 & XZ & Y = 0 & над пограничным слоем\\
				\hline
				PlaneXZ20M & XZ & Y = -20 & над препятствием\\
				\hline
				PlaneXZ23М & XZ & Y = -23 & на уровне проволоки\\
				\hline
			\end{tabular}
		\end{center}
		\label{tbl:sections}
		\caption{Перечень сечений}
	\end{table}
	
\section{Влияние на локальное трение и перенос}
	Далее для удобства описания результатов и их оценка, представлено несколько пунктов по различным параметрам.
\subsection{Скорость}
	Начнём анализ со скорости. По полученным сечениям были построены контуры средних скоростей для различных временных отрезках. Абсолютное значения скорости взято для каждого сечения своё, это позволяет лучше визуально отобразить изменения скоростей.
	\begin{figure}[H]
		\centering
		\includegraphics[width=1\linewidth]{../Assets/T16_Velocity_ContourXY}
		\caption{Скорость в продольном сечении XY, t = 1.6 c}
		\label{fig:t16velocitycontourxy}
	\end{figure}
	Как видно из рисунка вдоль канала, скорость сразу за препятствием и на его уровне приблизилась к нулю. А над проволокой значительно увеличилась. Из-за таких резких изменений скорости происходит образование вихревых структур, скорость которых хорошо заметна на этих изображениях. Эти вихревые образования можно заметить и на рисунке \ref{fig:T16VelocityContourYZ}. В сечении конца канала ($x = 1400$) скорость не изменилась, так как поток не достиг данного сечения.
	\begin{figure}[H]
		\begin{subfigure}{.5\textwidth}
			\centering
			\includegraphics[width=1.1\linewidth]{../Assets/T16_Velocity_ContourYZ340}
			\caption{PlaneYZ340}
			\label{fig:T16VelocityContourYZ340}
		\end{subfigure}%
		\begin{subfigure}{.5\textwidth}
			\centering
			\includegraphics[width=1.1\linewidth]{../Assets/T16_Velocity_ContourYZ400}
			\caption{PlaneYZ400}
			\label{fig:T16VelocityContourYZ400}
		\end{subfigure}
		\\
		\begin{subfigure}{.5\textwidth}
			\centering
			\includegraphics[width=1.1\linewidth]{../Assets/T16_Velocity_ContourYZ600}
			\caption{PlaneYZ600}
			\label{fig:T16VelocityContourYZ600}
		\end{subfigure}%
		\begin{subfigure}{.5\textwidth}
			\centering
			\includegraphics[width=1.1\linewidth]{../Assets/T16_Velocity_ContourYZ1400}
			\caption{PlaneYZ1400}
			\label{fig:T16VelocityContourYZ1400}
		\end{subfigure}
		\caption{Скорость в поперечных сечениях при t = 1.6 с}
		\label{fig:T16VelocityContourYZ}
	\end{figure}
	Через некоторое время основные турбулентные структуры наблюдаются в основном в пограничном слое.
	\begin{figure}[H]
		\centering
		\includegraphics[width=1\linewidth]{../Assets/T96_Velocity_ContourXY}
		\caption{Скорость в продольном сечении XY, t = 9.6 с}
		\label{fig:t96velocitycontourxy}
	\end{figure}
	К моменту времени t = 9.6 с вихревая структура канала стала менее выраженной и крупные вихри преобразовались в мелкие. 
	Рассмотрим подробнее скорость в поперечных сечениях в момент времени t = 9.6 с. Как видно на рисунке \ref{fig:T96VelocityContourYZ} образованные вихри постепенно распадаются до вихрей колмагоровского масштаба и диссипируют в энергию.
	\begin{figure}[H]
		\begin{subfigure}{.5\textwidth}
			\centering
			\includegraphics[width=1.1\linewidth]{../Assets/T96_Velocity_ContourYZ340}
			\caption{PlaneYZ340}
			\label{fig:T96VelocityContourYZ340}
		\end{subfigure}%
		\begin{subfigure}{.5\textwidth}
			\centering
			\includegraphics[width=1.1\linewidth]{../Assets/T96_Velocity_ContourYZ400}
			\caption{PlaneYZ400}
			\label{fig:T96VelocityContourYZ400}
		\end{subfigure}
		\\
		\begin{subfigure}{.5\textwidth}
			\centering
			\includegraphics[width=1.1\linewidth]{../Assets/T96_Velocity_ContourYZ600}
			\caption{PlaneYZ600}
			\label{fig:T96VelocityContourYZ600}
		\end{subfigure}%
		\begin{subfigure}{.5\textwidth}
			\centering
			\includegraphics[width=1.1\linewidth]{../Assets/T96_Velocity_ContourYZ1400}
			\caption{PlaneYZ1400}
			\label{fig:T96VelocityContourYZ1400}
		\end{subfigure}
		\caption{Скорость в поперечных сечениях при t = 9.6 с}
		\label{fig:T96VelocityContourYZ}
	\end{figure}
	\newpage
	На рисунке \ref{fig:T16VelocityContourXZ} мы можем наблюдать, как образуются вихревые структуры на разных высотах канала. Как видно, образованная структура достигла середины канала и выше. На высоте $y = -20$ мм ярко выражены вихри среднего размера, а уже на высоте $y = -23$ мм более мелкие структуры.
	\begin{figure}[H]
		\begin{subfigure}{1\textwidth}
			\centering
			\includegraphics[width=1\linewidth]{../Assets/T16_Velocity_ContourXZ0}
			\caption{PlaneXZ0}
			\label{fig:T16VelocityContourXZ0}
		\end{subfigure}%
		\\
		\begin{subfigure}{1\textwidth}
			\centering
			\includegraphics[width=1\linewidth]{../Assets/T16_Velocity_ContourXZ20M}
			\caption{PlaneXZ20M}
			\label{fig:T16VelocityContourXZ20M}
		\end{subfigure}%
		\\
		\begin{subfigure}{1\textwidth}
			\centering
			\includegraphics[width=1\linewidth]{../Assets/T16_Velocity_ContourXZ23M}
			\caption{PlaneXZ23M}
			\label{fig:T16VelocityContourXZ23M}
		\end{subfigure}
	 	\caption{Скорость в продольных сечениях XZ при t = 1.6 с}
	 	\label{fig:T16VelocityContourXZ}
	\end{figure}
	Некоторые другие временные промежутки и сечения представлены в приложении A.
	\newpage
\subsection{Коэффициент трения}
	Чтобы получить данные коэффициентов трения для построения графиков, в Ansys Fluent были построены необходимые сечения. Они делят канал вдоль на 3 части($z_1 = -31, z_2 = 0, z_3 = 31$). После этого встроенными средствами экспортировались данные в файлы формата $.csv$. Далее при помощи MS Excel были выделены данные касающиеся пограничного слоя. И в конце сгенерированы графики в gnuplot. Оценивались 4 временных отрезка($t_1 = 0.6, t_2 = 3.6, t_3 = 7.6, t_4 = 10.6$). На графиках отражена зависимость $C_f$ от координаты $x$. Далее рассмотрим три группы графиков.
	
	На первой группе изображений показаны изменения коэффициента трения в сечении при $z = 0$ мм. По прямолинейному участку на графике  можно понять до какого участка канала дошла жидкость в этот момент времени. При достижении потоком жидкости препятствия происходит резкое изменение $C_f$. По мере продвижения по длине канала коэффициент уменьшался. Это обусловлено уменьшением влияния вихревых структур на пограничный слой.
	\begin{figure}[H]
		\begin{subfigure}{.5\textwidth}
			\centering
			\includegraphics[width=1\linewidth]{../Assets/Cf-T06}
			\caption{t = 0.6 с}
			\label{fig:Cf-T06}
		\end{subfigure}%
		\begin{subfigure}{.5\textwidth}
			\centering
			\includegraphics[width=1\linewidth]{../Assets/Cf-T360}
			\caption{t = 3.6 с}
			\label{fig:Cf-T360}
		\end{subfigure}
		\\
		\begin{subfigure}{.5\textwidth}
			\centering
			\includegraphics[width=1\linewidth]{../Assets/Cf-T760}
			\caption{t = 7.6 с}
			\label{fig:Cf-T760}
		\end{subfigure}%
		\begin{subfigure}{.5\textwidth}
			\centering
			\includegraphics[width=1\linewidth]{../Assets/Cf-T1060}
			\caption{t = 10.6 с}
			\label{fig:Cf-T1060}
		\end{subfigure}
		\caption{Изменение коэффициента трения по длине канала, z = 0 мм}
		\label{fig:cf}
	\end{figure}
	Далее на второй группе изображений показаны изменения коэффициента трения в сечении при $z = -31$ мм. Здесь, в отличии от предыдущего сечения, наблюдаются более резкие изменения коэффициента за препятствием.
	\begin{figure}[H]
		\begin{subfigure}{.5\textwidth}
			\centering
			\includegraphics[width=1\linewidth]{../Assets/Cf-T06-31m}
			\caption{t = 0.6 с}
			\label{fig:Cf-T06-31m}
		\end{subfigure}%
		\begin{subfigure}{.5\textwidth}
			\centering
			\includegraphics[width=1\linewidth]{../Assets/Cf-T360-31m}
			\caption{t = 3.6 с}
			\label{fig:Cf-T360-31m}
		\end{subfigure}
		\\
		\begin{subfigure}{.5\textwidth}
			\centering
			\includegraphics[width=1\linewidth]{../Assets/Cf-T760-31m}
			\caption{t = 7.6 с}
			\label{fig:Cf-T760-31m}
		\end{subfigure}%
		\begin{subfigure}{.5\textwidth}
			\centering
			\includegraphics[width=1\linewidth]{../Assets/Cf-T1060-31m}
			\caption{t = 10.6 с}
			\label{fig:Cf-T1060-31m}
		\end{subfigure}
		\caption{Изменение коэффициента трения по длине канала, z = -31 мм}
		\label{fig:cf-31m}
	\end{figure}
	По прямой части канала коэффициент трения имел значение 0.33(рисунок \ref{fig:Cf-T06-31m}). По мере продвижения потока по каналу значение выросло до 0.55 к моменту времени $t = 7.6$ с при $x = 400$ мм.
	
	Пиковые значения изменения в коэффициенте трения наблюдаются в момент времени $t = 10.6$ с (рисунок \ref{fig:Cf-T1060-31m}). Значение выросло практически в 4.5 раза.
	\newpage
	В левой части канала, $z = 31$ мм, наблюдается такое же изменение, как и в правой. Тем не менее, это несколько меньше, чем в предыдущем случае.
	\begin{figure}[H]
		\begin{subfigure}{.5\textwidth}
			\centering
			\includegraphics[width=1\linewidth]{../Assets/Cf-T06-31p}
			\caption{t = 0.6 с}
			\label{fig:Cf-T06-31p}
		\end{subfigure}%
		\begin{subfigure}{.5\textwidth}
			\centering
			\includegraphics[width=1\linewidth]{../Assets/Cf-T360-31p}
			\caption{t = 3.6 с}
			\label{fig:Cf-T360-31p}
		\end{subfigure}
		\\
		\begin{subfigure}{.5\textwidth}
			\centering
			\includegraphics[width=1\linewidth]{../Assets/Cf-T760-31p}
			\caption{t = 7.6 с}
			\label{fig:Cf-T760-31p}
		\end{subfigure}%
		\begin{subfigure}{.5\textwidth}
			\centering
			\includegraphics[width=1\linewidth]{../Assets/Cf-T1060-31p}
			\caption{t = 10.6 с}
			\label{fig:Cf-T1060-31p}
		\end{subfigure}
		\caption{Изменение коэффициента трения по длине канала, z = 31 мм}
		\label{fig:cf-31p}
	\end{figure}
	Совмещённые варианты графиков представлены в приложении B. По ним можно установить, что обилие вихревых структур образовалось в правой части канала($z < 0$). 
\subsection{Q критерий}
	Существуют различные подходы к визуализации вихревых течений, использующие то или иное определение вихря и критерии его идентификации. Классификация методов визуализации вихревых течений проводится в зависимости от того, каким способом определяется вихрь (в области или на линии), является ли метод инвариантным по отношению к преобразованию системы координат, носит подход локальный или глобальный характер\cite{Hunt1988}. Одним из таких является $Q$ критерий($Q-criterion$). Он определяется по двум компонентам: тензор скоростей деформаций $S$ и тензор вращения $\Omega$. 
	\begin{equation}
		S = \frac{1}{2}(\frac{\partial u_i}{\partial x_j} + \frac{\partial u_j}{\partial x_i}) \qquad \Omega = \frac{1}{2}(\frac{\partial u_i}{\partial x_j} - \frac{\partial u_j}{\partial x_i})
	\end{equation}
	Критерий $Q$ определяется как второй инвариант тензора градиента скорости\cite{Wiebel2007}.
	\begin{equation}
		Q = \frac{1}{2}(||\Omega||^2 - ||S||^2)
	\end{equation}
	
	Из этого мы можем видеть, что положительные значения $Q$ указывают на области в поле течения, где преобладает завихренность, а отрицательные значения $Q$ указывают на области, в которых преобладают скорость деформации или вязкое напряжение. Кроме того, существуют различные варианты $Q$ критерия в модифицированных выражениях, которые используются для описания различных областей и структур течения\cite{Berdahl1993,Chong1990}.
	
	Различные области вихревых образований представлено ниже.
	\begin{figure}[H]
		\centering
		\includegraphics[width=1\linewidth]{../Assets/Q860-t16}
		\caption{Вихревая структура при Q = 860 и t = 1.6 c}
		\label{fig:q860-t16}
	\end{figure}
	На рисунке \ref{fig:q860-t16} можно заметить явно выраженную структуру образовавшегося вихря. 
	\begin{figure}[H]
		\centering
		\includegraphics[width=1\linewidth]{../Assets/QM850-t16}
		\caption{Вихревая структура при Q = -850 и t = 1.6 с}
		\label{fig:qm850-t16}
	\end{figure}
	При оценке значений $Q$ критерия, можно понять, что низкие значения показывают более мелкие вихревые структуры. С повышением значения $Q$ более крупные вихри можно обнаружить.
	\nocite{KimW.W.1995,Agostini2014,Blackwelder1983,Abbas2017,Лаптева2013,Мазо2007,Корнев2005}
