\section{Постановка задачи}
	Начальная скорость на входе в канал $U_0 = 0.261$ м/с.
	На выходе из канала выставлено условие равенства нулю производной по нормали к границе.
	\begin{equation}
		\frac{\partial}{\partial n} = 0
	\end{equation}
	Граничные условия непротекания и прилипания устанавливаются для стенки. Это выражено равенством нулю нормальной и тангенциальной составляющих скорости.
	\begin{equation}
		v \cdot n = 0 \qquad v \cdot \tau = 0
	\end{equation}
	Здесь $n$ и $\tau$ представляют собой единичные векторы нормали и касательной к поверхности канала. Граничные условия для давления выставляются при помощи дискретизации уравнения изменения количества движения в проекции на нормаль к стенке.

\section{Визуализация поставленной задачи}
\subsection{Геометрия канала}
	% Перефразировать или более красивое описание %
	Геометрическая модель представлена прямоугольным каналом с размером 369.5$\times$149.8 мм в начале. Далее он сужается на протяжении 396 мм до размеров 124$\times$50 мм. После он представляет собой прямую часть длинною 1100 мм. На расстоянии 323.9 мм от входа в канал внизу расположен вырез представляющий проволоку радиусом 2.1 мм и высотой 1.98 мм.
	\begin{figure}[H]
		\centering
		\includegraphics[width=0.7\linewidth]{../Assets/ВидКанала1}
		\caption{Общий вид канала}
		\label{fig:channelview}
	\end{figure}
	Сужение канала необходимо для увеличения скорости потока перед препятствием.
	\begin{figure}[H]
		\centering
		\includegraphics[width=0.6\linewidth]{../Assets/ВидКанала4}
		\caption{Вид препятствия в канале}
		\label{fig:vortexgeneratorview}
	\end{figure}

\subsection{Построение сеточной модели}
	% Описать построение сетки, способы и про оптимизацию + схема из отчёта + вся статистика сетки %
	Существует несколько методов моделирования сеточных моделей в Ansys, которые можно использовать в зависимости от типа и размера модели.
	
	Первый -- это метод пространственного разбиения, который часто используется для моделирования твердых тел. Этот метод заключается в разбиении объекта на более мелкие элементы, называемые конечными элементами. Затем каждый конечный элемент аппроксимируется более простыми формами, такими как треугольники или прямоугольники, чтобы создать сетку.
	
	Второй -- это метод генерации сетки на основе узлов. В этом методе модель представляется в виде набора узлов, соединенных линиями или поверхностями. Затем сетка строится на основе этой структуры.
	
	Третий -- это метод многократного разделения. Этот метод часто используется для моделирования пространственных объектов, таких как воздушные суда или автомобили. Он заключается в разбиении объекта на более мелкие блоки и последующем разделении каждого блока на еще более мелкие блоки. Затем каждый блок аппроксимируется более простыми формами для создания сетки.
	
	\begin{figure}[H]
		\centering
		\includegraphics[width=0.8\linewidth]{../Assets/СхемаСозданияСетки}
		\caption{Схема работы над сеточной моделью}
		\label{fig:meshScheme}
	\end{figure}
	
	На рисунке \ref{fig:meshScheme} представлена схематический план генерации сетки. Этот способ наиболее оптимальный для построения достаточно качественной сеточной модели. Разбиение геометрии на части позволяет ускорить построение, путём параллельного распределения вычислений(на каждую часть выделяется одно ядро). Кроме того, выключения режима многопоточности(одно физическое ядро делится на два виртуальных) для процессора увеличивает производительность, т. к. используется вся мощность ядра, а не его половина.
	
	% Перефразировать этот абзац и более красивее и обьёмнее + нужны хорошие скриншоты сетки %
	В результате работы над сеточной моделью удалось добиться оптимального результата для вычислений. Модель канала была разбита на 4 блока. Первый блок -- вход в канал, второй -- участок с препятствием, третий -- до конца сужения, четвёртый -- прямой участок канала. Статистика сетки: 51397337 узлов и 12665608 элементов. Полученная модель имеет уплотнение к низу канала, т. к. особую важность для изучения в данной работе составляет пограничный слой.

\section{Вычисление в ANSYS Fluent}
	% Тут про настройки fluent. Взять все параметры которые задавались при вычислении и оформить таблицей с описанием %
	ANSYS Fluent -- это программное обеспечение для численного моделирования физических процессов в жидкостях, газах и теплообменных устройствах. С помощью ANSYS Fluent можно проводить расчеты течения жидкости или газа, теплопередачи, химических реакций и других важных явлений. Программа поддерживает широкий спектр физических моделей и методов решения, что позволяет ее применять для самых разнообразных задач. ANSYS Fluent имеет удобный пользовательский интерфейс, который позволяет легко создавать, настраивать и запускать вычислительные модели. Также программа предоставляет мощные инструменты для анализа результатов и визуализации данных. В целом, ANSYS Fluent является одной из самых популярных и мощных программ для численного моделирования в области тепло- и массопереноса, гидродинамики и других областей физики.