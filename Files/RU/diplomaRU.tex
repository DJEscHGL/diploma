\documentclass[a4paper]{extreport}
\usepackage[14pt]{extsizes}
\usepackage{fontspec}
\defaultfontfeatures{Ligatures={TeX},Renderer=Basic} 
\setmainfont[Ligatures={TeX,Historic}]{Times New Roman}
\usepackage[english, russian]{babel}
\usepackage{setspace}
\usepackage{float}
\usepackage{indentfirst}
\usepackage{amsmath}
\usepackage{graphicx}
\usepackage{caption}
\usepackage{subcaption}
\usepackage{array}
\usepackage{titlesec}
\usepackage{appendix}
\usepackage[footnote]{acronym}
\usepackage[top=2cm, bottom=2cm, left=3cm, right=1cm, nofoot, nohead]{geometry}
\usepackage{titlesec}
\usepackage{tocloft}
\usepackage{lastpage}
\usepackage{etoolbox}

\titleformat{\chapter}[display]
{\filcenter}
{{\large\textbf{\MakeUppercase{\chaptertitlename} \thechapter}}}
{5pt}
{\bfseries}{}

\titleformat{\section}
{\normalsize\bfseries}
{\thesection}
{1em}{}

\titleformat{\subsection}
{\normalsize\bfseries}
{\thesubsection}
{1em}{}

% Настройка вертикальных и горизонтальных отступов
\titlespacing*{\chapter}{0pt}{-30pt}{8pt}
\titlespacing*{\section}{\parindent}{*4}{*1}
\titlespacing*{\subsection}{\parindent}{*4}{*1}

\newcommand{\likechapterheading}[1]{ 
	\begin{center}
		\textbf{\MakeUppercase{#1}}
	\end{center}
}

\makeatletter
	\renewcommand{\@dotsep}{2}
	\newcommand{\l@likechapter}[2]{{\@dottedtocline{0}{0pt}{0pt}{\bfseries #1}{\bfseries #2}}}
\makeatother

\newcommand{\likechapter}[1]{    
	\likechapterheading{#1}    
	\addcontentsline{toc}{likechapter}{\MakeUppercase{#1}}}

\renewcommand{\cfttoctitlefont}{\hspace{0.38\textwidth} \bfseries\MakeUppercase}
\renewcommand{\cftbeforetoctitleskip}{-1em}
\renewcommand{\cftchapfont}{\normalsize\bfseries \MakeUppercase{\chaptername} }
\renewcommand{\cftbeforechapskip}{1em}
\renewcommand{\cftparskip}{-1mm}
\renewcommand{\cftdotsep}{1}
\setcounter{tocdepth}{2}

\linespread{1.1}
\setlength{\parindent}{1.25cm}
\bibliographystyle{plain}
\renewcommand{\contentsname}{Оглавление}
\counterwithin{figure}{chapter}
\setlength{\footskip}{25pt}

% для en версии заменить russian на english %
\addto\captionsrussian{\renewcommand{\bibname}{}}
\patchcmd{\thebibliography}{\chapter*}{}{}{}
\patchcmd{\thebibliography}{\chapter*{\refname}}{}{}{}


\newcommand{\likechapterheeading}[1]{ 
	\begin{flushright}
		\textbf{\MakeUppercase{#1}}
	\end{flushright}
}

\makeatletter
\renewcommand{\@dotsep}{2}
\newcommand{\l@likeechapter}[2]{{\@dottedtocline{0}{0pt}{0pt}{\bfseries #1}{\bfseries #2}}}
\makeatother

\newcommand{\likeechapter}[1]{    
	\likechapterheeading{#1}    
	\addcontentsline{toc}{likechapter}{\MakeUppercase{#1}}}


\pretocmd{\chapter}{\addtocounter{totfigures}{\value{figure}}}{}{}
\pretocmd{\chapter}{\addtocounter{tottables}{\value{table}}}{}{}

\newcounter{totreferences}
\pretocmd{\bibitem}{\addtocounter{totreferences}{1}}{}{}

\newcounter{totfigures}
\newcounter{tottables}
\makeatletter
\AtEndDocument{%
	\addtocounter{totfigures}{\value{figure}}%
	\addtocounter{tottables}{\value{table}}%
	\immediate\write\@mainaux{%
		\string\gdef\string\totfig{\number\value{totfigures}}%
		\string\gdef\string\tottab{\number\value{tottables}}%
		\string\gdef\string\totref{\number\value{totreferences}}%    
	}%
}
\makeatother



\begin{document}
	\def\figurename{Рисунок}
\begin{titlepage}
	\begin{center}
		~~~
		\\
		~~~
		\\
		~~~
		\textbf{МИНИСТЕРСТВО ОБРАЗОВАНИЯ РЕСПУБЛИКИ БЕЛАРУСЬ\\
			БЕЛОРУССКИЙ ГОСУДАРСТВЕННЫЙ УНИВЕРСИТЕТ\\
			МЕХАНИКО-МАТЕМАТИЧЕСКИЙ ФАКУЛЬТЕТ\\
			Кафедра теоретической и прикладной механики\\}
		~~~
		\\
		~~~
		\\
		~~~
		\\
		~~~
		Шевелёв Дмитрий Юрьевич\\
		~~~
		\\
		~~~
		\textbf{ВЛИЯНИЕ ВИХРЕГЕНЕРАТОРОВ В ТУРБУЛЕНТНОМ ПОГРАНИЧНОМ СЛОЕ НА ЛОКАЛЬНОЕ ТРЕНИЕ И ПЕРЕНОС\\}
		~~~
		\\
		~~~
		\\
		~~~
		Дипломная работа\\
		~~~
		\\
		~~~
		\\
		~~~
	\end{center}
	\begin{flushright}
		Научный руководитель:\\
		кандидат физ.-мат. наук,\\
		доцент А. Д. Чорный\\
	\end{flushright}
	\begin{flushleft}
		Допущен к защите\\
		<<$\underline{\hspace{1cm}}$>>$\underline{\hspace{3cm}}$ 2023 г.\\
		Зав. кафедрой теоретической и прикладной механики\\
		доктор физ.-мат. наук, профессор М. А. Журавков
	\end{flushleft}
	~~~
	\\
	~~~
	\begin{center}
		Минск, 2023
	\end{center}
\end{titlepage}
\tableofcontents
\newpage
\likechapter{Перечень условных обозначений}
\begin{acronym}[RANS]
	\acro  {dns}  [DNS]   {Direct Numerical Solution, прямое численное моделирование}
	\acro  {rans} [RANS]  {Reynolds-Averaged Navier–Stokes, уравнения Навье-Стокса, осреднённые по Рейнольдсу}
	\acro  {les}  [LES]   {Large Eddy Simulation, метод моделирования крупных вихрей}
	\acro  {sgs}  [SGS]   {Sub Grid Scale, модели подсеточного масштаба}
	\acro  {des}  [DES]   {Detached Eddy Simulation, метод моделирования отсоединённых вихрей}
\end{acronym}
\newpage
\likechapter{Введение}

	Турбулентные пограничные слои развиваются на поверхностях многих инженерных конструкций: от теплообменных устройств, элементов воздушно-реактивных двигателей до планеров самолетов, корпусов кораблей и крупных строительных сооружений. Они определяют как сопротивление трения, так и перенос тепла. Вихревая структура этих слоев открывает возможность ее изменения путем воздействия на процесс формирования вихрей. 
	
	Одним из способов управления и уменьшения потерь энергии в турбулентном пограничном слое является использование методов активного и пассивного контроля турбулентности, например, установка поперечных ребер на поверхности или введение потока тепла в стенки. Более глубокое понимание особенностей турбулентного пограничного слоя может помочь снизить энергетические затраты и повысить эффективность различных технологий.
\newpage
	\chapter{Теория пограничного слоя и его моделирование}
	\section{Понятие пограничного слоя}
	Движение вязких сред связано с явлениями переноса в пограничном слое, где локализуются сопротивления трения, тепло- и массоотдачи. Понятие пограничного слоя впервые использовано Людвигом Прандтлем в статье, представленной 12 августа 1904 г., на третьем Международном конгрессе математиков в Гейдельберге, Германия. Пограничный слой -- область течения вязкой жидкости с малой по сравнению с продольными размерами поперечной толщиной, образующаяся у поверхности обтекаемого твёрдого тела или на границе раздела двух потоков жидкости с различными скоростями или температурами. Пограничный слой характеризуется резким изменением в поперечном направлении скорости (динамический пограничный слой) или температуры (температурный пограничный слой).
	
	Классическим примером пограничного слоя является пограничный слой, который образуется на плоской пластине при обтекании ее поверхности жидкостью и пограничный слой в круглых трубах. Более сложным для исследования и математического описания является пограничный слой на поверхностях с различной кривизной (обтекание цилиндра, сферы и др. тел). Такой пограничный слой характеризуется большим градиентом давления и точкой отрыва, за которой производная и скорость потока меняют знаки. Так же значительно сложны пограничные слои на поверхности раздела двухфазных и многофазных сред.
	
	Чем меньше вязкость среды, тем тоньше гидродинамический пограничный слой и большее значение в этом слое имеет градиент скорости. Вне пограничного слоя градиент скорости невелик. Следовательно, силы трения здесь малы, и ими обычно пренебрегают. Между внешним потоком и пограничным слоем резкой границы нет, поскольку средняя скорость жидкости по сечению потока изменяется монотонно, без скачков. Обычно толщину пограничного слоя определяют условно, исходя из того, что на его внешней границе скорость составляет 99 \% от скорости внешнего потока.
	
	Значение пограничного слоя очень велико, так как он определяет гидродинамическое сопротивление при движении среды относительно твердого тела, а также сопротивление переносу массы и тепла. Введение этого понятия существенно упростило уравнения моделирования течение жидкости путём разделения потока на две области.
	
	В общем случае неизотермическое движение вязкой сжимаемой жидкости описывается следующими уравнениями: уравнения Навье-Стокса, уравнение неразрывности, конвективной теплопроводности и состояния.
	\begin{align}
		& u\frac{\partial u}{\partial x} + v\frac{\partial u}{\partial y} + w\frac{\partial u}{\partial z} = -\frac{1}{\rho}\frac{\partial p}{\partial x} + \nu(\frac{\partial^2 u}{\partial x^2} + \frac{\partial^2 u}{\partial y^2} + \frac{\partial^2 u}{\partial z^2}) \nonumber\\
		& u\frac{\partial v}{\partial x} + v\frac{\partial v}{\partial y} + w\frac{\partial v}{\partial z} = -\frac{1}{\rho}\frac{\partial p}{\partial y} + \nu(\frac{\partial^2 v}{\partial x^2} + \frac{\partial^2 v}{\partial y^2} + \frac{\partial^2 v}{\partial z^2})\nonumber\\
		& u\frac{\partial w}{\partial x} + v\frac{\partial w}{\partial y} + w\frac{\partial w}{\partial z} = -\frac{1}{\rho}\frac{\partial p}{\partial z} + \nu(\frac{\partial^2 w}{\partial x^2} + \frac{\partial^2 w}{\partial y^2} + \frac{\partial^2 w}{\partial z^2})\nonumber\\
		& \frac{\partial u}{\partial x} + \frac{\partial v}{\partial y} + \frac{\partial w}{\partial z} = 0 \nonumber\\
		& \rho c (\frac{\partial T}{\partial\tau} + \vec{v}\nabla T) = div(\lambda grad(T)) + q_v + \mu\Phi - p div(\vec{v}) \nonumber\\
		& f(p, \rho, T) = 0
	\end{align}
	Исходя из этого имеем 6 уравнений и 6 неизвестных($u, v, w, \rho, T, p$). Эта система чрезвычайно сложна и в общем виде с трудом поддается решению даже современными численными методами на мощных компьютерах. Поэтому часто рассматривают различные упрощения этой системы.
	
	Существуют три вида течения в пограничном слое, каждое из которых имеет свои особенности и некоторые из них достаточны сложны для численного моделирования:
	\begin{itemize}
		\item ламинарное -- движение жидкости упорядочено, слои не смешиваются, частицы вращаются в пределах одного и того же тонкого слоя;
		\item турбулентное -- движение неупорядочено, происходит перемешивание частиц в поперечном направлении и весь пограничный слой беспорядочно завихрен;
		\item смешанное -- переходное состояние от ламинарного к турбулентному режиму движения.
	\end{itemize}
	В данной работе изучается турбулентный режим течения. Он является более сложным для численного моделирования.
\section{Турбулентное состояние пограничного слоя}
	Ламинарное течение устойчиво только при некоторых условиях, определяемых значением критического числа Рейнольдса $Re_{cr}$. Обычно переход от ламинарного к турбулентному режиму течения жидкости в трубах наблюдается при $Re_{cr} \approx 2300$. Турбулентное движение в пограничном слое возникает из-за нестабильности потока, которая проявляется в виде вихрей различных размеров и интенсивности. Эти вихри перемешивают слои жидкости, что приводит к увеличению переноса массы и энергии вдоль поверхности твердого тела. Турбулентное течение с большим числом Рейнольдса называют развитой турбулентностью.
	
	На значение критического числа можно влиять. Если вход в трубу сделать плавным, то ламинарное движение в трубе может поддерживаться при больших числах Рейнольдса, например до 24000. Существенно влияют на $Re_{cr}$ и такие факторы, как градиент давления, форма канала, шероховатость его стенок, вдув и откачка пограничного слоя. Возрастание давления в направлении движения приводит к неустойчивости течения в пограничном слое, отрыву и возникновению вихрей. Поэтому как для внутренних (в трубах и каналах), так и для внешних (обтекание тел потоком) течений критическое число Рейнольдса возрастает при уменьшении внешнего градиента давления (ускоряющиеся течения -- по закону Бернулли).
	\begin{figure}[H]
		\centering
		\includegraphics[width=0.7\linewidth]{../Assets/turb}
		\caption{Пример турбулентности}
		\label{fig:turb}
	\end{figure}

	Турбулентность можно определить как трехмерное нестационарное движение, в котором вследствие растяжения вихрей создается непрерывное распределение пульсаций скорости в интервале длин волн от минимальных, определяемых вязкими силами, до максимальных, определяемых граничным условиями течения. При математическом описании турбулентных течений удобно исходить из понимания турбулентности как иерархии вихрей различного масштаба, используя вихревую и волновую трактовку турбулентности\cite{Монин1992}. Турбулентные вихри непрерывны и постоянно соприкасаются друг с другом, причем крупные вихри, размеры которых определяются граничными условиями задачи, содержат в себе вихри меньших размеров.
	
	Максимальный размер вихрей близок к характерному линейному масштабу задачи $L$. Часто движение наиболее крупных вихрей оказывается в значительной степени упорядоченным (например, течение за цилиндром). Такие структуры называют когерентными. Вихри минимального размера диссипируют непосредственно в тепло. Их размер характеризуется колмогоровским масштабом $\eta_k = (\nu^3/\epsilon)^{1/4}$. Наибольшее количество энергии при этом переносят вихри некоторого среднего размера.
	
	Одной из важных характеристик описывающих турбулентное движение это завихрённость $\omega$: 
	\begin{align}
		\vec{\omega} = \vec{\nabla}\times \vec{v} \qquad \omega_x = \frac{\partial w}{\partial y} - \frac{\partial v}{\partial z}, \qquad \omega_y = \frac{\partial u}{\partial z} - \frac{\partial w}{\partial x}, \qquad \omega_z = \frac{\partial v}{\partial x} - \frac{\partial u}{\partial y}
	\end{align}

\section{Структура пограничного слоя}
	Представления о структуре профиля скорости постепенно менялись и окончательно сформировались к концу 1950х годов. В турбулентном пограничном слое обычно выделяется две области: внешняя и внутрення. Они отличаются друг от друга разными масштабами вихревых структур\cite{Белов2001}.
	\begin{figure}[H]
		\centering
		\includegraphics[width=0.7\linewidth]{../Assets/ПогранСлойRU}
		\caption{Схема слоя}
	\end{figure}
	Внутренняя область пограничного слоя занимает примерно 20\% от толщины всего слоя и в ней генерируется до 80\% энергии турбулентности. На формирование течения в пограничном слое основное влияние оказывают вязкость, теплопроводность и диффузионная способность жидкости. Внутри динамического пограничного слоя происходит плавное изменение скорости от её значения во внешнем потоке до нуля на стенке вследствие прилипания вязкой жидкости к твёрдой поверхности. Аналогично внутри пограничного слоя плавно изменяется температура.
			
\subsection{Внешняя область}

	Внешний слой является областью полностью развитого турбулентного течения, в котором распределение скорости описывается логарифмическим законом. Полное затухание возмущений во внешней области происходит на расстоянии, во много раз превышающем линейный масштаб турбулентности.
	
	Чтобы определить поток во внешней зоне, применяют фильтрованные или осредненные по Рейнольдсу уравнения Навье-Стокса. В то же время, профиль скорости во внутренней зоне сравнительно мало зависит от различных внешних условий, таких как числа Рейнольдса и градиент давления, что позволяет использовать универсальные соотношения (пристеночные функции) для связи параметров потока с расстоянием от стенки. Этот метод также базируется на гипотезе о локальном равновесии энергии турбулентных пульсаций и свойствах локальной изотропности диссипирующих вихрей\cite{Волков2005}.

\subsection{Внутренняя область}

	Вязкий подслой, буферный и логарифмический слои составляют внутреннюю область пограничного слоя. Она характеризуется высокой скоростью переноса массы, импульса и тепла, что приводит к повышенной трению и потере энергии.
	\begin{align}
		v\frac{\partial u}{\partial y} \gg -\overline{u'v'} & \qquad\text{вязкий}\nonumber \\
		v\frac{\partial u}{\partial y} \approx -\overline{u'v'} & \qquad\text{буферный}\nonumber \\
		v\frac{\partial u}{\partial y} \ll -\overline{u'v'} & \qquad\text{логарифмический}
	\end{align}
	
	Существует два подхода к моделированию течения в пристеночной области. В первом подходе используются полуэмпирические формулы (пристеночные функции) для описания внутреннего слоя потока, в то время как во втором подходе модели турбулентности модифицируются таким образом, чтобы разрешать всю пристеночную область потока, включая вязкий подслой, при условии обеспечения необходимого разрешения сетки в пограничном слое. Такие модели турбулентности могут быть использованы для расчета турбулентных течений во всей расчетной области (включая пристеночную область течения).

\subsection{Свойства пограничного слоя}
	
	Толщина пограничного слоя трудно определима как в расчёте, так и в эксперименте. Для определения используются понятия: толщина вытеснения $\delta^*$ и толщина потери импульса $\theta$.
	\begin{equation}
		\delta^* = \int_{0}^{\infty}(1 - \frac{u}{U_0})dy \qquad \theta = \delta^{**} = \int_{0}^{\infty} \frac{u}{U_0}(1 - \frac{u}{U_0})dy
	\end{equation}
	Кроме того используется безразмерный параметр $H$:
	\begin{equation}
		H = \frac{\delta^*}{\theta}
	\end{equation}
	
	Число Рейнольдса характеризуется двумя величинами($Re_x$ и $Re_\theta$): расстоянием от нижней стенки $x$ и толщиной $\theta$.
	\begin{equation}
		Re_x = \frac{xU_0}{\nu} \qquad Re_\theta = \frac{\theta U_0}{\nu}
	\end{equation}
	
	Используя напряжение трения на стенке $\tau_w$ можем вычислить коэффициент трения $C_F$ и динамическую скорость $u_\tau$:
	\begin{equation}
		\tau_w = \nu\frac{\partial u}{\partial y}\bigg|_W \qquad C_F = \frac{\tau_w}{0.5\rho U_0^2} \qquad u_\tau = \sqrt{\frac{\tau_w}{\rho}}
	\end{equation}
	
	Не менее важной характеристикой пограничных слоев является продольный градиент давления:
	\begin{equation}
		\frac{dp}{dx} = -\rho U_0 \frac{dU_0}{dx}
	\end{equation}
	
	Часто на пограничные слои влияют такие факторы: кривизна поверхности $\kappa$, скорость закачки и откачки жидкости, шероховатость поверхности $k_s^+$(высота бугорков).
	\begin{equation}
		\kappa = \frac{\delta^*}{R} \qquad \frac{V_W}{u_\tau}, \frac{V_W}{U_0} \qquad k_s^+ = \frac{k_s u_\tau}{\nu}
	\end{equation}
	
	Важным свойством пограничного слоя является выполнение интегрального уравнения импульсов. Верно и обратное: если уравнение импульсов не выполняется, то уравнения плоского пограничного слоя также не верны для этого течения. Это может быть обусловлено разными причинами: трехмерность течения, его нестационарность, влияние вверх по потоку, изменение давления поперек пограничного слоя, влияние нормальных Рейнольдсовых напряжений и т.д.
	\begin{equation}
		\frac{d\theta}{dx} + \frac{dU_0}{dx}\cdot\frac{2 + H}{U_0}\theta - \frac{C_f}{2} = 0
	\end{equation}

\subsection{Закон стенки}
	
	Для работы с пограничным слоем обычно используются безразмерные параметры $u^+$ и $y^+$:
	\begin{equation}
		u^+ = \frac{u}{u_\tau} \qquad y^+ = \frac{yu_\tau}{v}
	\end{equation}
	В вязком подслое вязкие напряжения намного больше над рейнольдсовыми, а скорость линейно зависит от расстояния до стенки. В буферном слое напряжения имеют одинаковый порядок. В логарифмическом слое рейнольдсовы напряжения намного превышают вязкие напряжения, а профиль скорости описывается логарифмическим законом.
	\begin{align}
		&u^+ = y^+ & 0 \leq y^+ < 5 & \qquad\text{вязкий} \nonumber\\
		&u^+ \neq y^+ & 5 \leq y^+ < 60 & \qquad\text{буферный} \nonumber\\
		&u^+ = \frac{1}{Ka}ln E y^+ & 60 \leq y^+ < 0.2\delta & \qquad\text{логарифмический}
	\end{align}
	В данном случае число Кармана $Ka$ и параметр $E$, который зависит от шероховатости поверхности, определяются на основе экспериментальных данных. 
	\begin{equation}
		Ka = \frac{1}{u}\sqrt{\frac{R_x^2 + R_y^2 + R_z^2}{3}}
	\end{equation}

	Закон стенки можно рассматривать как решение уравнений движения в турбулентном пограничном слое, полученное при использовании модели пути смешения Прандтля. Обычно считается, что закон стенки выполняется при $30 < y^+ < 200$. Конкретная реализация подхода зависит от выбранной модели турбулентности и используемой разностной схемы. Функции, входящие в закон стенки, нуждаются в модификации для точного описания эффектов, связанных с возможной шероховатостью поверхности. Закон стенки трудно приспособить для расчета течений со сложной геометрией на неструктурированных сетках.

\section{Методы моделирования}
	
	Несмотря на интенсивное развитие вычислительной техники и впечатляющие успехи, достигнутые в последние годы как в области построения эффективных численных алгоритмов, предназначенных для решения задач гидромеханики и тепломассопереноса, так и в разработке сопутствующего математического обеспечения (генераторы сеток, интерактивные системы ввода данных и систем визуализации результатов расчетов), проблема численного моделирования турбулентности, как и на протяжении многих предшествующих десятилетий, по-прежнему остается одной из наиболее сложных и актуальных проблем механики жидкостей. В отличие от ламинарных течений однофазной среды (жидкости), расчет которых, благодаря отмеченным выше достижениям, стал во многом рутинной процедурой, надежное предсказание характеристик сложных турбулентных течений, представляющих наибольший практический интерес все еще остается сложным.
	\begin{figure}[H]
		\centering
		\includegraphics[width=0.9\linewidth]{../Assets/СхемаМетодовRU}
		\caption{Виды методов по уравнениям}
		\label{fig:cheme}
	\end{figure}
	
	Среди основных методов численного моделирования трехмерных турбулентных течений можно выделить: прямое численное моделирование (DNS), моделирование крупных вихрей (LES) и решение осредненных по Рейнольдсу уравнений Навье-Стокса (RANS). Имеются также различные промежуточные подходы, сочетающие в себе те или иные черты RANS, LES и DNS, например, метод моделирования отсоединенных вихрей (DES), и ряд других, не имеющих должного физического обоснования и потому не получивших широкого распространения.

\subsection{DNS}
	
	Прямое численное моделирование(DNS) предполагает решение полных нестационарных трехмерных уравнений Навье-Стокса, позволяющее получить мгновенные характеристики турбулентного потока. Проблемы с широким использованием DNS связаны с высокими требованиями к используемой разностной схеме, удовлетворением начальных и граничных условий, а также ограниченными ресурсами вычислительной техники. Расчетная область при этом должна быть достаточно протяженной, чтобы вместить наибольшие масштабы турбулентности, а шаг интегрирования по времени должен иметь порядок колмогоровского масштаба.
	
	\begin{table}[H]
		\begin{center}
			\begin{tabular}{|c|c|c|c|c|}
				\hline
				Re & 6.6$\times10^3$ & 2.0$\times10^4$ & 1.0$\times10^5$ & 1.0$\times10^6$\\
				\hline
				Кол-во узлов сетки & 2$\times10^6$ & 4$\times10^7$ & 3$\times10^8$ & 1.5$\times10^3$\\
				\hline
				150 MFlops & 37 ч & 740 ч & 6.5 лет & 3000 лет\\
				\hline
				1 TFlops & 20 с & 400 с & 8.3 ч & 4000 ч\\
				\hline
			\end{tabular}
		\end{center}
		\caption{Затраты времени при различных параметрах}
	\end{table}

	Эти жесткие требования отчасти смягчаются при использовании высокоточных спектральных методов численного интегрирования уравнений Навье-Стокса, которые	часто используются для DNS. Однако эти методы неприменимы к расчету течений со сложной геометрией. Указанные обстоятельства приводят к тому, что на практике DNS применяется только для расчета простых турбулентных течений при низких числах Рейнольдса. При этом основной задачей расчета является не	собственно получение данных о характеристиках осредненного течения (они, как правило, известны), а получение детальной информации о структуре турбулентности, а также вычисление отдельных членов, входящих в те или иные модели турбулентности.

\subsection{RANS}
	
	В инженерных приложениях широко используются математические модели, основанные на численном решении осредненных по Рейнольдсу уравнений Навье-Стокса(RANS). При использовании уравнений Рейнольдса основной интерес проявляется к динамике крупномасштабных вихрей, ответственных за переносные свойства турбулентных течений. При замыкании уравнений Рейнольдса рассматриваются масштабы длины, типичные для энергосодержащих вихрей, в которых $Re\gg1$ (за исключением пристеночных областей). Для учета пристеночного влияния диссипирующих вихрей и энергосодержащих вихрей при $Re\sim1$ используются демпфирующие функции. Применив к уравнениям осреднение Рейнольдса получим:
	\begin{align}
				&\frac{\partial u_i}{\partial x_i} = 0 \nonumber\\
				&\rho\frac{\partial u_i}{\partial t} + \rho u_j \frac{\partial u_i}{\partial x_j} = - \frac{\partial p}{\partial x_i} + \frac{\partial}{\partial x_j}(\mu\frac{\partial u_i}{\partial x_j} - \rho\overline{u_i' u_j'})
	\end{align}

	Эта система является незамкнутой, поскольку в нее входит неизвестный тензор так называемых рейнольдсовых напряжений $\tau_{ij} = -\rho\overline{u_i' u_j'}$(турбулентные напряжения). Для замыкания этой системы уравнений необходимо определить шесть различных компонент симметричного тензора турбулентных напряжений. Именно выражение этих компонент через параметры осредненного потока и называется моделью турбулентности. Ниже представлена таблица с основными этапами развития теории.
	
	\begin{table}[H]
		\begin{center}
			\begin{tabular}{|c|c|c|}
				\hline
				Год & Ученые & Что изучено\\
				\hline
				1877 & Буссинеск Ж. В. & гипотеза Буссинеска\\
				\hline
				1895 & Рейнольдс О. & осреднение по Рейнольдсу\\
				\hline
				1925 & Прандтль Л. & теория пути смешивания Прандтля\\
				\hline
				1930 & Карман Т. & формула Кармана\\
				\hline
				1942 & Колмогоров А. Н. & формула Колмогорова, модель $k$-$\omega$\\
				\hline
				1951 & Ротта & первая модель Рейнольдсовых напряжений\\
				\hline
				1956 & Клаузер & формула Клаузера\\
				\hline
				1956 & Ван-Дрист & демпфирующий множитель\\
				\hline
				1974 & Лондер Б. и Сполдинг Д. & модель $k$-$\epsilon$\\
				\hline
			\end{tabular}
		\end{center}
		\caption{Этапы развития теории}
	\end{table}
	
	Появление огромного количества моделей привело к необходимости выбора. Для этого необходимо провести сравнительный анализ моделей. Однако при попытке тестирования естественным образом возникают определенные трудности. Во-первых, необходимо выбрать течения, для которых известен набор достоверных экспериментальных данных, свободных от погрешностей, а также выбрать критерии для сравнения моделей. Во-вторых, необходимо провести серийные расчеты этих течений с использованием разных моделей турбулентности и при этом быть уверенными в независимости результата от программной реализации задачи. Результатом такой работы должны явиться рекомендации по области применимости тех или иных моделей турбулентности.

\subsection{LES}
	
	Метод моделирования крупных вихрей(LES) был предложен Иосифом Смагоринским в 1963 году. Он основан на двух предположениях. Первый предполагает, что течение можно разделить на движение крупных и мелких вихрей. Крупные вихри, находящиеся под прямым воздействием граничных условий и несущие в себе максимум рейнольдсовых напряжений, рассчитываются. Мелкомасштабная турбулентность считается изотропной и имеющей универсальные характеристики, а потому менее критичной и более поддающейся моделированию. Другой  заключается в возможности аппроксимации нелинейных взаимодействий между крупными и мелкими вихрями только по крупным вихрям с использованием подсеточных моделей(SGS). Иначе говоря, принимается гипотеза о статистической независимости крупных и мелких вихрей.
	
	Статистика крупных вихрей обычно не чувствительна к подсеточному моделированию за исключением пристеночной области. Имеющиеся подсеточные модели корректно предсказывают не только осредненные характеристики потока (первые и вторые моменты), но также и флуктуации интегральных характеристик, например, коэффициентов сопротивления и подъемной силы\cite{Fureby2000}.
	
	Мелкомасштабное движение исключается из уравнений Навье-Стокса при помощи применения операции фильтрации и моделируются с помощью подсеточных моделей. На рисунке \ref{fig:lesfilter} показан принцип работы фильтров, где $g(x)$ - исходный вариант, $f(x)$ - после фильтрации.\\
	\begin{figure}[H]
		\centering
		\includegraphics[width=0.7\linewidth]{../Assets/ФильтрацияLES}
		\caption{Исключение мелкомасштабных движений фильтрацией}
		\label{fig:lesfilter}
	\end{figure}
	Уравнение фильтра, применимое к пространственно временному полю $\phi(x,t)$ представлено ниже:
	\begin{equation}
		\overline{\phi(x,t)} = \int_{-\infty}^{\infty}\int_{-\infty}^{\infty}\phi(r,t')G(x - r, t - t')dt'dr
	\end{equation}
	В данном случае $G$ - ядро, характерное для каждого типа фильтра.
	
	Решение, полученное с помощью LES, содержит более богатую информацию по сравнению с решением на основе уравнений Рейнольдса, например, не только характеристики среднего течения (поля скорости, концентрации, температуры, давления) и распределения рейнольдсовых напряжений, но также и спектральные характеристики (спектры пульсаций скорости и давления), двухточечные моменты, временные и пространственные масштабы турбулентности.

\subsection{DES}
	
	Характерные для отрывных течений крупномасштабные нестационарные трехмерные вихревые структуры определяются конкретными граничными условиями и геометрическими характеристиками рассматриваемых течений и не могут быть описаны в рамках таких моделей. Это стимулируют поиск и разработку гибридных подходов, сочетающих в себе экономичность RANS и универсальность LES.
	
	В методе моделирования отсоединенных вихрей (DES) в области присоединенного пограничного слоя метод функционирует в режиме уравнений Рейнольдса, а в области отрыва потока переходит в LES. При этом достигается сочетание лучших качеств обоих подходов -- высокая точность и экономичность уравнений Рейнольдса в области присоединенного пограничного слоя и универсальность LES в отрывной области. Хотя DES, в отличие от RANS, является принципиально нестационарным трехмерным подходом, необходимые для его реализации сетки в пристеночной области совпадают с сетками, необходимыми для решения уравнений Рейнольдса, и являются на много порядков меньшими, чем сетки, требуемые для разрешения мелких пристенных вихрей в рамках LES. По мере измельчения сетки DES асимптотически приближается к LES и далее к DNS. Конкретные реализации DES основаны на использовании модели турбулентной вязкости Спаларта-Аллмараса и модели Ментера\cite{Strelets2001}.
	
	Как следует из названия метода DES, он создавался для расчета отрывных течений. Такие течения лучше подходят для этого метода. Во-первых, наличие массированного отрыва в большинстве случаев приводит к его пульсациям, как следствие этого, к возникновению автоколебательного течения с крупными когерентными структурами. Во-вторых, наличие отрывной зоны позволяет обойти проблему создания турбулентных пульсаций на входе в LES области.
\subsection{Оценка производительности}
	
	Оценка количества узлов сетки и временных шагов, необходимых для реализации DNS и LES, показывает сложность проблемы с вычислительной точки зрения.
	
	\begin{table}[H]
		\begin{center}
			\begin{tabular}{|c|c|c|c|c|}
				\hline
				Метод & Число узлов сетки & Число шагов по времени & Готовность\\
				\hline
				RANS & $10^7$ & $10^3$ & 1985\\
				\hline
				DES & $10^8$ & $10^4$ & 2000\\
				\hline
				LES & $10^{11.5}$ & $10^{6.7}$ & 2045\\
				\hline
				DNS & $10^{16}$ & $10^{7.7}$ & 2080\\
				\hline
			\end{tabular}
		\end{center}
		\caption{Перспектива применения методов}
	\end{table}
	Готовность означает практическое применение метода с затратой времени не более суток.
	Для оценки необходимых вычислительных ресурсов (например, быстродействия и объема оперативной памяти) возьмем расчетную сетку 100$\times$100$\times$100 узлов($10^6$ точек). В каждом узле необходимо вычислить около 10 функций (составляющие скорости, плотность, давление, температуру, характеристики турбулентности, концентрации компонентов смеси). Значения неизвестных функций находятся в результате решения системы нелинейных уравнений, что требует от 200 до 1000 арифметических операций. За один шаг по времени необходимо выполнить $10^{10}$ операций с плавающей точкой. Для исследования развития процесса во времени требуется до 1000 временных шагов. В результате, выполнение одного расчета требует $10^{13}$ операций с плавающей точкой. Для проведения одного расчетного варианта компьютер с производительностью 100 МФлопc($10^8$ операций с плавающей точкой в секунду) затратит $10^7$ секунд. Для проведения расчета за 100 минут потребуется компьютер с производительностью 0.1 ТФлопс.
	\chapter{Моделирование с использованием ANSYS FLUENT}
	\section{Постановка задачи}
	Начальная скорость на входе в канал $U_0 = 0.29$ м/с. Интенсивность турбулентности 5\%. На выходе из канала выставлено условие равенства нулю производной по нормали к границе. Граничные условия непротекания и прилипания устанавливаются для стенки. Это выражено равенством нулю нормальной и тангенциальной составляющих скорости.
	\begin{equation}
		v \cdot n = 0 \qquad v \cdot \tau = 0
	\end{equation}
	Здесь $n$ и $\tau$ представляют собой единичные векторы нормали и касательной к поверхности канала. Граничные условия для давления выставляются при помощи дискретизации уравнения изменения количества движения в проекции на нормаль к стенке.
	
	Стенка сдерживает рост мелких вихрей и изменяет механизм обмена энергией между разрешимыми и неразрешимыми масштабами турбулентности. При этом число сеточных узлов, необходимых для расчета течения в пограничном слое, возрастает до величины, характерной для DNS. С целью уменьшения потребления вычислительных ресурсов и учета влияния различных факторов, например, шероховатости поверхности, используются метод пристеночных функций и различные модели турбулентного пограничного слоя\cite{Cabot2000}.
	
	Для нестационарных расчетов применяется условие конвективного переноса (неотражающие граничные условия). Поток массы на входе равняется потоку массы на выходе из расчетной области.
	\begin{equation}
		\frac{\partial f}{\partial t} + U_0\frac{\partial f}{\partial n} = 0
	\end{equation}

\section{Визуализация поставленной задачи}
	Для построения геометрии канала и создания сеточной модели использовались встроенные средства программного обеспечения Ansys. Он предоставляет универсальное, высокопроизводительное, автоматизированное, интеллектуальное решение для построения сеточной модели, которое создает наиболее подходящую модель для точных и эффективных физических решений -- от простого автоматического построения сетки до сложной, более гибридной и детальной сетки.
\subsection{Геометрия канала}
	Объектом исследования является турбулентный пограничный слой в канале. Канал разбит на две части. Первая часть представляет собой сужение от 369.5$\times$149.8 мм в начале до размеров 124$\times$50 мм. Длина этого участка 396 мм. Он позволяет значительно увеличить скорость потока жидкости. Вторая часть -- прямолинейная, длиною в 1100 мм.
	\begin{figure}[H]
		\centering
		\includegraphics[width=0.7\linewidth]{../Assets/ВидКанала1}
		\caption{Общий вид канала}
		\label{fig:channelview}
	\end{figure}
	На расстоянии 323.9 мм от входа в канал, у основания расположен вырез представляющий проволоку радиусом 2.1 мм и высотой 1.98 мм. Это препятствие и создаёт турбулентное состояние. 
	\begin{figure}[H]
		\centering
		\includegraphics[width=0.6\linewidth]{../Assets/ВидКанала4}
		\caption{Вид препятствия в канале}
		\label{fig:vortexgeneratorview}
	\end{figure}

\subsection{Построение сеточной модели}
	
	Наиболее важной частью любого численного моделирования -- это создание достаточно подробной сеточной модели. От ее качества зависит точность полученных результатов. Кроме того она требует более мелкого разбиения для пограничного слоя, нежели чем основная часть канала, т. к. основные процессы образования вихрей происходит именно там.
	
	\begin{figure}[H]
		\centering
		\includegraphics[width=0.7\linewidth]{../Assets/СхемаСозданияСеткиRU}
		\caption{Схема работы над сеточной моделью}
		\label{fig:meshScheme}
	\end{figure}
	
	На рисунке \ref{fig:meshScheme} представлен схематический план генерации сетки. Этот способ наиболее оптимальный для построения достаточно качественной сеточной модели. Разбиение геометрии на части позволяет ускорить построение, путём параллельного распределения вычислений(на каждую часть выделяется одно ядро). Кроме того, выключения режима многопоточности(одно физическое ядро делится на два виртуальных) для процессора увеличивает производительность, т. к. используется вся мощность ядра, а не его половина.
	
	Существует несколько методов моделирования сеточных моделей в Ansys, которые можно использовать в зависимости от типа и размера модели.
	
	Первый -- это метод пространственного разбиения, который часто используется для моделирования твердых тел. Этот метод заключается в разбиении объекта на более мелкие элементы, называемые конечными элементами. Затем каждый конечный элемент аппроксимируется более простыми формами, такими как треугольники или прямоугольники, чтобы создать сетку.
	
	Второй -- это метод генерации сетки на основе узлов. В этом методе модель представляется в виде набора узлов, соединенных линиями или поверхностями. Затем сетка строится на основе этой структуры.
	
	Третий -- это метод многократного разделения. Этот метод часто используется для моделирования пространственных объектов, таких как воздушные суда или автомобили. Он заключается в разбиении объекта на более мелкие блоки и последующем разделении каждого блока на еще более мелкие блоки. Затем каждый блок аппроксимируется более простыми формами для создания сетки.
	
	\begin{figure}[H]
		\centering
		\includegraphics[width=1\linewidth]{../Assets/Mesh1}
		\caption{Сеточная модель}
		\label{fig:mesh1}
	\end{figure}
	
	Используя упомянутые выше план и способы построения, была построена сеточная модель. С этой сеткой удалось добиться оптимального результата для вычислений. Канал был поделён на 4 части. Первая часть -- вход в канал, вторая -- участок с препятствием, третья -- до конца сужения, четвёртая -- прямой участок канала. Сеточная модель состоит из 51397337 узлов и 12665608 элементов.
	
\section{Расчёт задачи}
	Ansys Fluent -- это программное обеспечение для численного моделирования физических процессов в жидкостях, газах и теплообменных устройствах. С помощью Ansys Fluent можно проводить расчеты течения жидкости или газа, теплопередачи, химических реакций и других важных явлений. Программа поддерживает широкий спектр физических моделей и методов решения, что позволяет ее применять для самых разнообразных задач. Ansys Fluent имеет удобный пользовательский интерфейс, который позволяет легко создавать, настраивать и запускать вычислительные модели. Также программа предоставляет мощные инструменты для анализа результатов и визуализации данных. В целом, Ansys Fluent является одной из самых популярных и мощных программ для численного моделирования в области тепло- и массопереноса, гидродинамики и других областей физики.
	
	Перейдём к подготовке к вычислениям. В качестве жидкости использовалась вода c характеристиками: $\rho = 998.2$ $kg/m^3$ и $\nu = 0.001003$ $kg/m\cdot s$. Для подсеточной модели метода LES использовалась модель WALE c коэффициентом $C_w = 0.325$. Основные преимущества данной модели:
	\begin{itemize}
		\item пространственный оператор содержит как локальные деформации, так и скорости вращения. Таким образом, все структуры турбулентности, имеющие отношение к диссипации кинетической энергии, вычисляются этой моделью;
		\item турбулентная вязкость стремится к нулю вблизи стенки, так что ни постоянная(динамическая) регулировка, ни функция демпфирования не требуются для расчета течений, ограниченных стенкой;
		\item модель дает нулевую турбулентную вязкость при чистом сдвиге. Таким образом, он может воспроизвести процесс перехода от ламинарного к турбулентному потоку за счет роста линейных неустойчивых режимов. 
	\end{itemize}
	Кроме того, модель WALE инвариантна к любому перемещению или вращению координат, и требуется только локальная информация (отсутствие операции проверки-фильтрации и сведений о ближайших точках в сетке), так что она хорошо подходит для LES в сложных геометриях\cite{Nicoud1999}.
	
	В качестве расчётной схемы использовался алгоритм SIMPLEC. Он представляет из себя полу-неявный метод для согласованных уравнений, связанных с давлением. Эта схема является модифицированной формой SIMPLE метода. Его алгоритм был разработан Ван Дормалом и Райтби в 1984 году. В нём за счет изменения определения коэффициентов уравнения поправки на давление частично компенсируются эффекты отбрасывания соседних сеточных поправок на скорость (второе приближение в алгоритме SIMPLE)\cite{Sun2008}.
	
	Некоторые другие параметры, связанные с расчётом уравнений:
	\begin{table}[H]
		\begin{center}
			\begin{tabular}{|c|c|}
				\hline
				Параметр & Метод\\
				\hline
				Градиент & Least squares cell based\\
				\hline
				Давление & Second order\\
				\hline
				Импульс & Bounded central differencing\\
				\hline
				Время & Bounded second order implicit\\
				\hline
			\end{tabular}
		\end{center}
		\caption{Перечень параметров решателя}
	\end{table}
	
	С описанными настройками и параметрами выше, был запущен расчёт в ANSYS FLUENT. Размер временного шага $\Delta t = 0.001 s$, их количество $s_t = 10000$. Это составляет 10 с реального времени. На каждый шаг рассчитывалось $s_i = 50$ итераций.
	\chapter{Анализ полученных данных}
	\section{Обработка данных}
	В результате вычислений, которые заняли примерно месяц, были полученные $.dat$ файлы. Каждый из этих массивов данных соответствует определённому временному шагу. 
	
	После окончания вычислений был автоматически сгенерирован график, благодаря которому можно оценить точность полученных результатов на данной сеточной модели. Она достигла значения в $10^{-3}$.
	\begin{figure}[H]
		\centering
		\includegraphics[width=1\linewidth]{../Assets/scaledResiduals}
		\caption{График сходимости}
		\label{fig:scaledresiduals}
	\end{figure}
	
	Для последующей обработки и на их основе построение графиков использовалось встроенное средство Ansys CFD-Post. Это программа для обработки результатов расчетов, которая обладает широким набором всех необходимых инструментов: создание анимаций, графиков, контурных распределений параметров потока, векторных полей, линий тока, объемного рендеринга, отчетов по шаблонам и многого другого. Для анализа влияния вихрегенераторов на локальное трение и перенос использовались различные сечения в разных частях канала. Некоторые из них представлены на рисунке \ref{fig:planesforanalysis}. 
	\begin{figure}[H]
		\centering
		\includegraphics[width=0.9\linewidth]{../Assets/1}
		\caption{Сечения для анализа}
		\label{fig:planesforanalysis}
	\end{figure}
	
	Начало координатных осей модели расположено в центре входа канала. Рассматривались сечения на разных высотах вдоль и различных расстояниях от входа в канал. На таблице ниже представлен перечень сечений, расположение и их описание. Изображения в приложениях имеют названия исходя из таблицы.
	\begin{table}[H]
		\begin{center}
			\begin{tabular}{|c|c|c|c|}
				\hline
				Подпись & Плоскость & Расположение, мм & Описание\\
				\hline
				PlaneXY & XY & Z = 0 & вдоль всего канала\\
				\hline
				PlaneYZ300 & YZ & X = 300 & перед препятствием\\
				\hline
				PlaneYZ340 & YZ & X = 340 & за проволокой\\
				\hline
				PlaneYZ400 & YZ & X = 400 & на входе в прямой участок\\
				\hline
				PlaneYZ600 & YZ & X = 600 & далее по каналу\\
				\hline
				PlaneYZ1400 & YZ & X = 1400 & на выходе из канала\\
				\hline
				PlaneXZ0 & XZ & Y = 0 & над пограничным слоем\\
				\hline
				PlaneXZ20M & XZ & Y = -20 & над препятствием\\
				\hline
				PlaneXZ23М & XZ & Y = -23 & на уровне проволоки\\
				\hline
			\end{tabular}
		\end{center}
		\label{tbl:sections}
		\caption{Перечень сечений}
	\end{table}
	
\section{Влияние на локальное трение и перенос}
	Далее для удобства описания результатов и их оценка, представлено несколько пунктов по различным параметрам.
\subsection{Скорость}
	Начнём анализ со скорости. По полученным сечениям были построены контуры средних скоростей для различных временных отрезках. Абсолютное значения скорости взято для каждого сечения своё, это позволяет лучше визуально отобразить изменения скоростей.
	\begin{figure}[H]
		\centering
		\includegraphics[width=1\linewidth]{../Assets/T16_Velocity_ContourXY}
		\caption{Скорость в продольном сечении XY, t = 1.6 c}
		\label{fig:t16velocitycontourxy}
	\end{figure}
	Как видно из рисунка вдоль канала, скорость сразу за препятствием и на его уровне приблизилась к нулю. А над проволокой значительно увеличилась. Из-за таких резких изменений скорости происходит образование вихревых структур, скорость которых хорошо заметна на этих изображениях. Эти вихревые образования можно заметить и на рисунке \ref{fig:T16VelocityContourYZ}. В сечении конца канала ($x = 1400$) скорость не изменилась, так как поток не достиг данного сечения.
	\begin{figure}[H]
		\begin{subfigure}{.5\textwidth}
			\centering
			\includegraphics[width=1.1\linewidth]{../Assets/T16_Velocity_ContourYZ340}
			\caption{PlaneYZ340}
			\label{fig:T16VelocityContourYZ340}
		\end{subfigure}%
		\begin{subfigure}{.5\textwidth}
			\centering
			\includegraphics[width=1.1\linewidth]{../Assets/T16_Velocity_ContourYZ400}
			\caption{PlaneYZ400}
			\label{fig:T16VelocityContourYZ400}
		\end{subfigure}
		\\
		\begin{subfigure}{.5\textwidth}
			\centering
			\includegraphics[width=1.1\linewidth]{../Assets/T16_Velocity_ContourYZ600}
			\caption{PlaneYZ600}
			\label{fig:T16VelocityContourYZ600}
		\end{subfigure}%
		\begin{subfigure}{.5\textwidth}
			\centering
			\includegraphics[width=1.1\linewidth]{../Assets/T16_Velocity_ContourYZ1400}
			\caption{PlaneYZ1400}
			\label{fig:T16VelocityContourYZ1400}
		\end{subfigure}
		\caption{Скорость в поперечных сечениях при t = 1.6 с}
		\label{fig:T16VelocityContourYZ}
	\end{figure}
	Через некоторое время основные турбулентные структуры наблюдаются в основном в пограничном слое.
	\begin{figure}[H]
		\centering
		\includegraphics[width=1\linewidth]{../Assets/T96_Velocity_ContourXY}
		\caption{Скорость в продольном сечении XY, t = 9.6 с}
		\label{fig:t96velocitycontourxy}
	\end{figure}
	К моменту времени t = 9.6 с вихревая структура канала стала менее выраженной и крупные вихри преобразовались в мелкие. 
	Рассмотрим подробнее скорость в поперечных сечениях в момент времени t = 9.6 с. Как видно на рисунке \ref{fig:T96VelocityContourYZ} образованные вихри постепенно распадаются до вихрей колмагоровского масштаба и диссипируют в энергию.
	\begin{figure}[H]
		\begin{subfigure}{.5\textwidth}
			\centering
			\includegraphics[width=1.1\linewidth]{../Assets/T96_Velocity_ContourYZ340}
			\caption{PlaneYZ340}
			\label{fig:T96VelocityContourYZ340}
		\end{subfigure}%
		\begin{subfigure}{.5\textwidth}
			\centering
			\includegraphics[width=1.1\linewidth]{../Assets/T96_Velocity_ContourYZ400}
			\caption{PlaneYZ400}
			\label{fig:T96VelocityContourYZ400}
		\end{subfigure}
		\\
		\begin{subfigure}{.5\textwidth}
			\centering
			\includegraphics[width=1.1\linewidth]{../Assets/T96_Velocity_ContourYZ600}
			\caption{PlaneYZ600}
			\label{fig:T96VelocityContourYZ600}
		\end{subfigure}%
		\begin{subfigure}{.5\textwidth}
			\centering
			\includegraphics[width=1.1\linewidth]{../Assets/T96_Velocity_ContourYZ1400}
			\caption{PlaneYZ1400}
			\label{fig:T96VelocityContourYZ1400}
		\end{subfigure}
		\caption{Скорость в поперечных сечениях при t = 9.6 с}
		\label{fig:T96VelocityContourYZ}
	\end{figure}
	\newpage
	На рисунке \ref{fig:T16VelocityContourXZ} мы можем наблюдать, как образуются вихревые структуры на разных высотах канала. Как видно, образованная структура достигла середины канала и выше. На высоте $y = -20$ мм ярко выражены вихри среднего размера, а уже на высоте $y = -23$ мм более мелкие структуры.
	\begin{figure}[H]
		\begin{subfigure}{1\textwidth}
			\centering
			\includegraphics[width=1\linewidth]{../Assets/T16_Velocity_ContourXZ0}
			\caption{PlaneXZ0}
			\label{fig:T16VelocityContourXZ0}
		\end{subfigure}%
		\\
		\begin{subfigure}{1\textwidth}
			\centering
			\includegraphics[width=1\linewidth]{../Assets/T16_Velocity_ContourXZ20M}
			\caption{PlaneXZ20M}
			\label{fig:T16VelocityContourXZ20M}
		\end{subfigure}%
		\\
		\begin{subfigure}{1\textwidth}
			\centering
			\includegraphics[width=1\linewidth]{../Assets/T16_Velocity_ContourXZ23M}
			\caption{PlaneXZ23M}
			\label{fig:T16VelocityContourXZ23M}
		\end{subfigure}
	 	\caption{Скорость в продольных сечениях XZ при t = 1.6 с}
	 	\label{fig:T16VelocityContourXZ}
	\end{figure}
	Некоторые другие временные промежутки и сечения представлены в приложении A.
	\newpage
\subsection{Коэффициент трения}
	Чтобы получить данные коэффициентов трения для построения графиков, в Ansys Fluent были построены необходимые сечения. Они делят канал вдоль на 3 части($z_1 = -31, z_2 = 0, z_3 = 31$). После этого встроенными средствами экспортировались данные в файлы формата $.csv$. Далее при помощи MS Excel были выделены данные касающиеся пограничного слоя. И в конце сгенерированы графики в gnuplot. Оценивались 4 временных отрезка($t_1 = 0.6, t_2 = 3.6, t_3 = 7.6, t_4 = 10.6$). На графиках отражена зависимость $C_f$ от координаты $x$. Далее рассмотрим три группы графиков.
	
	На первой группе изображений показаны изменения коэффициента трения в сечении при $z = 0$ мм. По прямолинейному участку на графике  можно понять до какого участка канала дошла жидкость в этот момент времени. При достижении потоком жидкости препятствия происходит резкое изменение $C_f$. По мере продвижения по длине канала коэффициент уменьшался. Это обусловлено уменьшением влияния вихревых структур на пограничный слой.
	\begin{figure}[H]
		\begin{subfigure}{.5\textwidth}
			\centering
			\includegraphics[width=1\linewidth]{../Assets/Cf-T06}
			\caption{t = 0.6 с}
			\label{fig:Cf-T06}
		\end{subfigure}%
		\begin{subfigure}{.5\textwidth}
			\centering
			\includegraphics[width=1\linewidth]{../Assets/Cf-T360}
			\caption{t = 3.6 с}
			\label{fig:Cf-T360}
		\end{subfigure}
		\\
		\begin{subfigure}{.5\textwidth}
			\centering
			\includegraphics[width=1\linewidth]{../Assets/Cf-T760}
			\caption{t = 7.6 с}
			\label{fig:Cf-T760}
		\end{subfigure}%
		\begin{subfigure}{.5\textwidth}
			\centering
			\includegraphics[width=1\linewidth]{../Assets/Cf-T1060}
			\caption{t = 10.6 с}
			\label{fig:Cf-T1060}
		\end{subfigure}
		\caption{Изменение коэффициента трения по длине канала, z = 0 мм}
		\label{fig:cf}
	\end{figure}
	Далее на второй группе изображений показаны изменения коэффициента трения в сечении при $z = -31$ мм. Здесь, в отличии от предыдущего сечения, наблюдаются более резкие изменения коэффициента за препятствием.
	\begin{figure}[H]
		\begin{subfigure}{.5\textwidth}
			\centering
			\includegraphics[width=1\linewidth]{../Assets/Cf-T06-31m}
			\caption{t = 0.6 с}
			\label{fig:Cf-T06-31m}
		\end{subfigure}%
		\begin{subfigure}{.5\textwidth}
			\centering
			\includegraphics[width=1\linewidth]{../Assets/Cf-T360-31m}
			\caption{t = 3.6 с}
			\label{fig:Cf-T360-31m}
		\end{subfigure}
		\\
		\begin{subfigure}{.5\textwidth}
			\centering
			\includegraphics[width=1\linewidth]{../Assets/Cf-T760-31m}
			\caption{t = 7.6 с}
			\label{fig:Cf-T760-31m}
		\end{subfigure}%
		\begin{subfigure}{.5\textwidth}
			\centering
			\includegraphics[width=1\linewidth]{../Assets/Cf-T1060-31m}
			\caption{t = 10.6 с}
			\label{fig:Cf-T1060-31m}
		\end{subfigure}
		\caption{Изменение коэффициента трения по длине канала, z = -31 мм}
		\label{fig:cf-31m}
	\end{figure}
	По прямой части канала коэффициент трения имел значение 0.33(рисунок \ref{fig:Cf-T06-31m}). По мере продвижения потока по каналу значение выросло до 0.55 к моменту времени $t = 7.6$ с при $x = 400$ мм.
	
	Пиковые значения изменения в коэффициенте трения наблюдаются в момент времени $t = 10.6$ с (рисунок \ref{fig:Cf-T1060-31m}). Значение выросло практически в 4.5 раза.
	\newpage
	В левой части канала, $z = 31$ мм, наблюдается такое же изменение, как и в правой. Тем не менее, это несколько меньше, чем в предыдущем случае.
	\begin{figure}[H]
		\begin{subfigure}{.5\textwidth}
			\centering
			\includegraphics[width=1\linewidth]{../Assets/Cf-T06-31p}
			\caption{t = 0.6 с}
			\label{fig:Cf-T06-31p}
		\end{subfigure}%
		\begin{subfigure}{.5\textwidth}
			\centering
			\includegraphics[width=1\linewidth]{../Assets/Cf-T360-31p}
			\caption{t = 3.6 с}
			\label{fig:Cf-T360-31p}
		\end{subfigure}
		\\
		\begin{subfigure}{.5\textwidth}
			\centering
			\includegraphics[width=1\linewidth]{../Assets/Cf-T760-31p}
			\caption{t = 7.6 с}
			\label{fig:Cf-T760-31p}
		\end{subfigure}%
		\begin{subfigure}{.5\textwidth}
			\centering
			\includegraphics[width=1\linewidth]{../Assets/Cf-T1060-31p}
			\caption{t = 10.6 с}
			\label{fig:Cf-T1060-31p}
		\end{subfigure}
		\caption{Изменение коэффициента трения по длине канала, z = 31 мм}
		\label{fig:cf-31p}
	\end{figure}
	Совмещённые варианты графиков представлены в приложении B. По ним можно установить, что обилие вихревых структур образовалось в правой части канала($z < 0$). 
\subsection{Q критерий}
	Существуют различные подходы к визуализации вихревых течений, использующие то или иное определение вихря и критерии его идентификации. Классификация методов визуализации вихревых течений проводится в зависимости от того, каким способом определяется вихрь (в области или на линии), является ли метод инвариантным по отношению к преобразованию системы координат, носит подход локальный или глобальный характер\cite{Hunt1988}. Одним из таких является $Q$ критерий($Q-criterion$). Он определяется по двум компонентам: тензор скоростей деформаций $S$ и тензор вращения $\Omega$. 
	\begin{equation}
		S = \frac{1}{2}(\frac{\partial u_i}{\partial x_j} + \frac{\partial u_j}{\partial x_i}) \qquad \Omega = \frac{1}{2}(\frac{\partial u_i}{\partial x_j} - \frac{\partial u_j}{\partial x_i})
	\end{equation}
	Критерий $Q$ определяется как второй инвариант тензора градиента скорости\cite{Wiebel2007}.
	\begin{equation}
		Q = \frac{1}{2}(||\Omega||^2 - ||S||^2)
	\end{equation}
	
	Из этого мы можем видеть, что положительные значения $Q$ указывают на области в поле течения, где преобладает завихренность, а отрицательные значения $Q$ указывают на области, в которых преобладают скорость деформации или вязкое напряжение. Кроме того, существуют различные варианты $Q$ критерия в модифицированных выражениях, которые используются для описания различных областей и структур течения\cite{Berdahl1993,Chong1990}.
	
	Различные области вихревых образований представлено ниже.
	\begin{figure}[H]
		\centering
		\includegraphics[width=1\linewidth]{../Assets/Q860-t16}
		\caption{Вихревая структура при Q = 860 и t = 1.6 c}
		\label{fig:q860-t16}
	\end{figure}
	На рисунке \ref{fig:q860-t16} можно заметить явно выраженную структуру образовавшегося вихря. 
	\begin{figure}[H]
		\centering
		\includegraphics[width=1\linewidth]{../Assets/QM850-t16}
		\caption{Вихревая структура при Q = -850 и t = 1.6 с}
		\label{fig:qm850-t16}
	\end{figure}
	При оценке значений $Q$ критерия, можно понять, что низкие значения показывают более мелкие вихревые структуры. С повышением значения $Q$ более крупные вихри можно обнаружить.
	\nocite{KimW.W.1995,Agostini2014,Blackwelder1983,Abbas2017,Лаптева2013,Мазо2007,Корнев2005}

	\newpage
	\likechapter{Заключение}
	
В данной работе изучены различные методы моделирования турбулентных течений, их преимущества и недостатки, а также оценена производительность. Помимо этого изучены способы построения сеточной модели и создана оптимальная схема по работе над ней.

В результате проведённого моделирования канала в Ansys Fluent методом моделирования крупных вихрей было оценено влияние на пограничный слой. Их наличие повлияло на скоростные характеристика потока за препятствием. Благодаря размещению вихрегенератора поперёк образовался турбулентный режим движения. Эти вихревые структуры значительно повлияли на локальное трение и перенос. Коэффициент трения достиг значения в 4.5 раза. Но уже после вихрегенератора коэффициент снижался, стремясь к значениям до изменения в канале.

В последующих исследованиях возможно изучение с другими условиями и вариациями вихрегенераторов.
	\newpage
	\likechapter{Список использованных источников}
	\bibliography{bibliography}
	\newpage
	\anonsection{Приложение А}
\appendix
\counterwithout{figure}{section}
\setcounter{figure}{0}
\begin{figure}[H]
	\centering
	\includegraphics[width=0.9\linewidth]{../Assets/T0_Velocity_ContourXY}
	\caption{PlaneXY, t = 0.6 c}
	\label{fig:t0velocitycontourxy}
\end{figure}
\begin{figure}[H]
	\begin{subfigure}{.5\textwidth}
		\centering
		\includegraphics[width=1.7\linewidth, angle=90]{../Assets/T0_Velocity_ContourXZ20M}
		\caption{PlaneXZ20M}
		\label{fig:t0velocitycontourxz20m}
	\end{subfigure}%
	\begin{subfigure}{.5\textwidth}
		\centering
		\includegraphics[width=1.7\linewidth, angle=90]{../Assets/T0_Velocity_ContourXZ23M}
		\caption{PlaneXZ23M}
		\label{fig:t0velocitycontourxz23m}
	\end{subfigure}
		\caption{PlaneXZ, t = 0.6 c}
		\label{fig:t0velocitycontourxz}
\end{figure}
\newpage
\begin{flushright}
	\MakeUppercase{\textbf{Приложение А}}
\end{flushright}
\begin{figure}[H]
	\centering
	\includegraphics[width=0.9\linewidth]{../Assets/T1060_Velocity_ContourXY}
	\caption{PlaneXY, t = 10.6 c}
	\label{fig:t1060velocitycontourxy}
\end{figure}
\begin{figure}[H]
	\begin{subfigure}{.5\textwidth}
		\centering
		\includegraphics[width=1.7\linewidth, angle=90]{../Assets/T1060_Velocity_ContourXZ20M}
		\caption{PlaneXZ20M}
		\label{fig:t1060velocitycontourxz20m}
	\end{subfigure}%
	\begin{subfigure}{.5\textwidth}
		\centering
		\includegraphics[width=1.7\linewidth, angle=90]{../Assets/T1060_Velocity_ContourXZ23M}
		\caption{PlaneXZ23M}
		\label{fig:t1060velocitycontourxz23m}
	\end{subfigure}
	\caption{PlaneXZ, t = 10.6 c}
	\label{fig:t1060velocitycontourxz}
\end{figure}
	\newpage
	\likeechapter{Приложение B}
\appendix
\counterwithout{figure}{chapter}
\setcounter{figure}{0}
\begin{figure}[H]
	\centering
	\includegraphics[width=0.9\linewidth]{../Assets/Cf-Tall}
	\caption{Изменение коэффициента трения по длине канала}
	\label{fig:cf-tall}
\end{figure}
\newpage
\begin{flushright}
	\MakeUppercase{\textbf{Приложение B}}
\end{flushright}

\end{document}