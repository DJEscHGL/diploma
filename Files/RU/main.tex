\def\figurename{Рисунок}
\begin{titlepage}
	\begin{center}
		~~~
		\\
		~~~
		\\
		~~~
		\textbf{МИНИСТЕРСТВО ОБРАЗОВАНИЯ РЕСПУБЛИКИ БЕЛАРУСЬ\\
			БЕЛОРУССКИЙ ГОСУДАРСТВЕННЫЙ УНИВЕРСИТЕТ\\
			МЕХАНИКО-МАТЕМАТИЧЕСКИЙ ФАКУЛЬТЕТ\\
			Кафедра теоретической и прикладной механики\\}
		~~~
		\\
		~~~
		\\
		~~~
		\\
		~~~
		Шевелёв Дмитрий Юрьевич\\
		~~~
		\\
		~~~
		\\
		~~~
		\\
		~~~
		\textbf{ВЛИЯНИЕ ВИХРЕГЕНЕРАТОРОВ В ТУРБУЛЕНТНОМ ПОГРАНИЧНОМ СЛОЕ НА ЛОКАЛЬНОЕ ТРЕНИЕ И ПЕРЕНОС\\}
		~~~
		\\
		~~~
		\\
		~~~
		\\
		~~~
		Дипломная работа\\
		~~~
		\\
		~~~
		\\
		~~~
		\\
		~~~
	\end{center}
	\begin{flushright}
		Научный руководитель:\\
		кандидат физ.-мат. наук,\\
		доцент А. Д. Чорный\\
	\end{flushright}
	\begin{flushleft}
		Допущен к защите\\
		<<$\underline{\hspace{1cm}}$>>$\underline{\hspace{3cm}}$ 2023 г.\\
		Зав. кафедрой теоретической и прикладной механики\\
		доктор физ.-мат. наук, профессор М. А. Журавков
	\end{flushleft}
	~~~
	\\
	~~~
	\\
	~~~
	\\
	~~~
	\\
	~~~
	\\
	~~~
	\begin{center}
		Минск, 2023
	\end{center}
\end{titlepage}
\newpage
\tableofcontents
\newpage
\begin{center}
	\textbf{\MakeUppercase{реферат}}
\end{center}
% xx заменить и дополнить про область применения %
\textbf{Отчёт по дипломной работе:} \pageref*{LastPage}~с., \totfig~рис., \tottab~табл., \totref~источников.\\
\textbf{Ключевые слова:} \MakeUppercase{вихрегенератор, пограничый слой, турбулентность, метод моделирования крупных вихрей, локальное трение и перенос, сеточная модель, моделирование, анализ}.\\
\textbf{Объект исследования:} течение в канале с установленным вихрегенератором.\\
\textbf{Цель работы:} исследовать влияние вихрегенераторов на локальное трение и перенос.\\
\textbf{Методы исследование:} численное моделирование методом крупных вихрей.\\
\textbf{Результат:} оценка влияния вихрегенераторов на локальное трение и перенос.\\
\textbf{Область применения:} .\\
\newpage
\begin{center}
	\textbf{\MakeUppercase{рэферат}}
\end{center}
\textbf{Справаздача па дыпломнай працы:} \pageref*{LastPage}~с., \totfig~малюнкаў, \tottab~табліц., \totref~крыніц.\\
\textbf{Ключавыя словы:} \MakeUppercase{віхрэгенератар, памежны пласт, турбулентнасць, метад мадэлявання буйных віхур, лакальнае трэнне і перанос, сеткавая мадэль, мадэляванне, аналіз}.\\
\textbf{Аб'ект даследавання:} плынь у канале з усталяваным віхрэгенератарам.\\
\textbf{Мэта працы:} даследаваць уплыў віхрэгенератараў на лакальнае трэнне і перанос.\\
\textbf{Метады даследавання:} лікавае мадэляванне метадам буйных віхур.\\
\textbf{Вынік:} ацэнка ўплыву віхрэгенератараў на лакальнае трэнне і перанос.\\
\textbf{Вобласць прымянення:} .\\
\newpage
\begin{center}
	\textbf{\MakeUppercase{abstract}}
\end{center}
\textbf{Diploma:} \pageref*{LastPage}~p., \totfig~pictures, \tottab~tables, \totref~references.\\
\textbf{Keywords:} \MakeUppercase{vortex generator, large eddy simulation, boundary layer, turbulence, local friction and transport, grid model, modeling, analysis}.\\
\textbf{Object of research:} flow in channel with vortex generator.\\
\textbf{Purpose of research:} to research influence of vortex generator on local friction and transport.\\
\textbf{Research methods:} large eddy simulation\\
\textbf{Result:} assessment of the impact on local friction and transport\\
\textbf{Scope:} .\\
\newpage
\likechapter{Перечень условных обозначений}
\begin{acronym}[RANS]
	\acro  {dns}   [DNS]   {Direct Numerical Solution, прямое численное моделирование}
	\acro  {rans}  [RANS]  {Reynolds-Averaged Navier–Stokes, уравнения Навье-Стокса, осреднённые по Рейнольдсу}
	\acro  {les}   [LES]   {Large Eddy Simulation, метод моделирования крупных вихрей}
	\acro  {sgs}   [SGS]   {Sub Grid Scale, модели подсеточного масштаба}
	\acro  {des}   [DES]   {Detached Eddy Simulation, метод моделирования отсоединённых вихрей}
	\acro  {wale}  [WALE]  {Wall-adapting local eddy-viscosity, адаптирующаяся к стене локальная турбулентная вязкость}
	\acro  {dstar} [$\boldsymbol{\delta^*}$] {толщина вытеснения}
	\acro  {theta} [$\boldsymbol{\theta}$] {толщина потери импульса}
	\acro  {tauom} [$\boldsymbol{\tau_\omega}$] {напряжение трения на стенки}
	\acro  {cfric} [$\boldsymbol{C_F}$] {коэффициентр трения}
	\acro  {utauu} [$\boldsymbol{u_\tau}$] {динамическая скорость}
	\acro  {omega} [$\boldsymbol{\omega}$] {завихрённость}
	\acro  {uzero} [$\boldsymbol{U_0}$] {начальная скорость потока}
	\acro  {lbig}  [$\boldsymbol{L}$] {характерный масштаб}
	\acro  {uzero} [$\boldsymbol{\eta_k}$] {колмогоровский масштаб}
	\acro  {qcrit} [$\boldsymbol{Q}$] {критерий Q}
	\acro  {recri} [$\boldsymbol{Re_{cr}}$] {критическое число Рейнольдса}
\end{acronym}
\newpage
\likechapter{Введение}
	
	‍Многие тела и конструкции имеют турбулентные пограничные слои на своих поверхностях. Это относится к теплообменным устройствам, элементам воздушно-реактивных двигателей, планерам самолетов, корпусам кораблей и крупным строительным сооружениям. Такие слои влияют на сопротивление трения и передачу тепла. Вихревая структура этих слоев может быть изменена путем контроля процесса формирования вихрей.
	
	Одним из способов управления и уменьшения потерь энергии в турбулентном пограничном слое является использование методов активного и пассивного контроля турбулентности, например, установка поперечных ребер (вихрегенераторов) на поверхности или введение потока тепла в стенки. Более глубокое понимание особенностей турбулентного пограничного слоя может помочь снизить энергетические затраты и повысить эффективность различных технологий.
	
	Теоретические методы моделирования и исследования явлений переноса в пограничном слое условно можно классифицировать на точные, асимптотические, численные и приближенные. Для математического моделирования явлений переноса и определения эффективности проводимых процессов в промышленных аппаратах чаще используются приближенные и численные методы.
	
	Современное развитие турбулентности как науки не представляется возможным без применения мощных компьютеров, реализующих различные модели и позволяющие выявить их сильные и слабые стороны. Именно вычислительный эксперимент сегодня является источником развития моделей турбулентности, которые, в свою очередь, являются основой для создания новых вычислительных средств.
	
	Целью данной работы является исследования влияния вихрегенератора, установленного на всю длину канала, на локальное трение и перенос. Для проведения вычислений использовался пакет ANSYS Fluent и для постобработки полученных данных CFD Post. В качестве метода использовался метод моделирования крупных вихрей.
\newpage