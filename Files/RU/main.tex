\def\figurename{Рисунок}
\begin{titlepage}
	\begin{center}
		~~~
		\\
		~~~
		\\
		~~~
		\textbf{МИНИСТЕРСТВО ОБРАЗОВАНИЯ РЕСПУБЛИКИ БЕЛАРУСЬ\\
			БЕЛОРУССКИЙ ГОСУДАРСТВЕННЫЙ УНИВЕРСИТЕТ\\
			МЕХАНИКО-МАТЕМАТИЧЕСКИЙ ФАКУЛЬТЕТ\\
			Кафедра теоретической и прикладной механики\\}
		~~~
		\\
		~~~
		\\
		~~~
		\\
		~~~
		Шевелёв Дмитрий Юрьевич\\
		~~~
		\\
		~~~
		\textbf{ВЛИЯНИЕ ВИХРЕГЕНЕРАТОРОВ В ТУРБУЛЕНТНОМ ПОГРАНИЧНОМ СЛОЕ НА ЛОКАЛЬНОЕ ТРЕНИЕ И ПЕРЕНОС\\}
		~~~
		\\
		~~~
		\\
		~~~
		Дипломная работа\\
		~~~
		\\
		~~~
		\\
		~~~
	\end{center}
	\begin{flushright}
		Научный руководитель:\\
		кандидат физ.-мат. наук,\\
		доцент А. Д. Чорный\\
	\end{flushright}
	\begin{flushleft}
		Допущен к защите\\
		<<$\underline{\hspace{1cm}}$>>$\underline{\hspace{3cm}}$ 2023 г.\\
		Зав. кафедрой теоретической и прикладной механики\\
		доктор физ.-мат. наук, профессор М. А. Журавков
	\end{flushleft}
	~~~
	\\
	~~~
	\begin{center}
		Минск, 2023
	\end{center}
\end{titlepage}
\tableofcontents
\newpage
\likechapter{Перечень условных обозначений}
\begin{acronym}[RANS]
	\acro  {dns}  [DNS]   {Direct Numerical Solution, прямое численное моделирование}
	\acro  {rans} [RANS]  {Reynolds-Averaged Navier–Stokes, уравнения Навье-Стокса, осреднённые по Рейнольдсу}
	\acro  {les}  [LES]   {Large Eddy Simulation, метод моделирования крупных вихрей}
	\acro  {sgs}  [SGS]   {Sub Grid Scale, модели подсеточного масштаба}
	\acro  {des}  [DES]   {Detached Eddy Simulation, метод моделирования отсоединённых вихрей}
\end{acronym}
\newpage
\likechapter{Введение}

	Турбулентные пограничные слои развиваются на поверхностях многих инженерных конструкций: от теплообменных устройств, элементов воздушно-реактивных двигателей до планеров самолетов, корпусов кораблей и крупных строительных сооружений. Они определяют как сопротивление трения, так и перенос тепла. Вихревая структура этих слоев открывает возможность ее изменения путем воздействия на процесс формирования вихрей. 
	
	Одним из способов управления и уменьшения потерь энергии в турбулентном пограничном слое является использование методов активного и пассивного контроля турбулентности, например, установка поперечных ребер на поверхности или введение потока тепла в стенки. Более глубокое понимание особенностей турбулентного пограничного слоя может помочь снизить энергетические затраты и повысить эффективность различных технологий.
\newpage